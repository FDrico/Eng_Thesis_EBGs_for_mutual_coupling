A mis directores de tesis, el Dr. Ing. Walter Gustavo Fano y Mg. Ing. Silvina Boggi, por brindarme la posibilidad de trabajar con ellos, por la dedicación y por la paciencia que me tuvieron durante el desarrollo de este trabajo.

A Franco Spadavecchia, Santiago Caracciolo, Ivan Pollitzer y Francisco López Destain por prestarme sus oídos y sentarse a pensar conmigo algunos de los problemas que tuve que resolver.

Al Centro de Comunicación Científica de la Facultad de Ciencias Exactas y Naturales de la Universidad de Buenos Aires por cederme amablemente recursos de cómputo para que pudiera realizar las simulaciones del presente trabajo.

Al Profesor Ing. Guillermo Santiago, por haberme mostrado la elegancia del electromagnetismo, y sin cuyas clases de física nunca hubiera elegido Ingeniería Electrónica.

A la universidad pública, gratuita y de acceso irrestricto, que me brindó esta oportunidad.

A mi familia, por haberme acompañado en el largo camino que elegí, a pesar del esfuerzo y las dificultades que trajo aparejados que tuviera que vivir tan lejos.