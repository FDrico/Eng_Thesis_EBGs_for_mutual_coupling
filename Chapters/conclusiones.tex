%%%%%%%%%%%%%%%%
%%  Capítulo 6: Conclusiones  %%
%%%%%%%%%%%%%%%%


El difundido uso de la tecnología \textit{microstrip} suele ser explicado en base a los bajos costos de fabricación que presenta y a la facilidad para la implantación de componentes circuitales discretos. Sin embargo, para las altas frecuencias representa tanto una herramienta como un desafío: la falta de homogeneidad del medio por el que se propagan los campos electromagnéticos, y la consecuente susceptibilidad a la aparición de modos indeseados obligan a que todo análisis cabal que se realice sobre ellas resulte incompleto y en general, complejo. En muchos casos, la disminución del ancho del sustrato dieléctrico interviniente disminuye sustancialmente los efectos de segundo orden, aunque a costa de un resentimiento en la rigidez estructural.

Los sustratos \textit{microstrip} soportan la aparición de ondas de superficie por sus meras características constructivas, dado que presentan una impedancia de superficie reactiva. Estas ondas de superficie pueden ser controladas mediante diversas técnicas, como la efectiva división del sustrato y plano de tierra, las estructuras de plano de tierra degenerado (DGS) y el uso de \enquote{escalones} de permitividad eléctrica en las inmediaciones de los elementos que son fuente de ondas de superficie. En este trabajo se analizó el uso de estructuras EBG para lograr resultados similares. Como estas estructuras surgieron en el campo de la óptica, si bien se parte de conceptos básicos de electromagnetismo, la nomenclatura resulta, por el momento, ambigua, la información se encuentra dispersa y, debido a la relativa novedad de la aplicación de estructuras EBG en microondas, en muchos casos es también contradictoria.

En la mayoría de los casos la obtención de resultados analíticos resulta imposible, por lo que las técnicas de simulación numérica pueden resultar útiles para la predicción del comportamiento de las estructuras \textit{microstrip}. Sin embargo, la relativa complejidad de las mismas genera que las simulaciones de onda completa requieran de mucho tiempo hasta entregar los resultados esperados, por lo que se dificulta su utilización para el diseño. Debe tenerse en cuenta, además, que el diseño de antenas requiere de la consideración de la estructura EBG a utilizar, y que la misma no puede agregarse a radiadores prediseñados, debido a que la modificación del comportamiento de las ondas de superficie genera importantes efectos sobre las características de radiación de la antena.

En este trabajo se presentaron dos alternativas al uso de estas simulaciones de onda completa: el modelado circuital y la utilización de TLM bidimensional. En ninguno de los dos casos, sin embargo, se predijo satisfactoriamente el comportamiento frente a las ondas de superficie: Sólo se modeló el comportamiento ante excitaciones conducidas, en vistas de lograr, a futuro, un modelo que permita comenzar el cálculo a partir del uso de una fuente distribuida sobre los bordes de la estructura. En el caso de la simulación mediante TLM, dado que parte de conceptos intuitivos sencillos, la misma podría resultar una herramienta para la comprensión de los fenómenos más complejos. Sin embargo, dado que la implementación actual se realizó utilizando técnicas de programación ineficientes y un lenguaje de prototipado, los tiempos de simulación no se redujeron lo suficiente, por lo que aun no podrían ser usados en etapas de diseño.

Resulta importante destacar que las ondas de superficie no son la única fuente de acoplamiento y que, en general, no son la más importante. El uso de estructuras EBG debe supeditarse a la relación entre todas las causas de acoplamiento posibles, especialmente debido a que el diseño de estas estructuras es, por el momento, iterativo. La necesidad de iterar en el proceso de diseño, sumado a que las celdas unitarias que ofrecen mejores resultados son también las más complejas, ofrecen muchos grados de libertad y por lo tanto, una mayor cantidad de variables a considerar, dificulta y ralentiza la tarea.

Por otro lado, si bien no se realizaron mediciones para este trabajo, las mismas resultan un paso fundamental para la validación de los resultados analíticos y numéricos obtenidos. La bibliografía y los trabajos de investigadores en los últimos años han demostrado, mediante mediciones, que es posible utilizar estas estructuras para disminuir el acoplamiento mutuo entre antenas que comparten sustrato.

\section{Propuestas de trabajos futuros}

Dado que el presente trabajo representa un primer paso en la comprensión de los fenómenos asociados al funcionamiento de las estructuras EBG, las propuestas de trabajos futuros son muy variadas. En primer lugar, resulta menester intentar aplicar los conceptos de modelado por circuitos de parámetros concentrados para predecir el comportamiento ante la incidencia de ondas de superficie que lo excitan en forma distribuida. Además, para que las simulaciones por TLM resulten efectivamente útiles al momento de diseñar, es necesario volcar los conceptos a un lenguaje de programación tipado y más eficiente que Python. Finalmente, el diseño completo de un conjunto de antenas, y no el meramente ilustrativo aplicado en este trabajo, para probar, mediante simulaciones y, principalmente, mediciones, el funcionamiento de estas estructuras, permitiría avanzar sobre terreno más firme en el uso de EBGs con estos fines.

Por otro lado, además de las geometrías analizadas, en los últimos años han surgido diversas estructuras que permiten efectos multibanda, a partir de lograr una controlada variación geometría de las celdas unitarias, como se analiza en \cite{Kern:multiband}. Por otro lado, resulta importante lograr efectos similares utilizando celdas unitarias más pequeñas, de modo que se pueda ubicar una mayor cantidad de celdas entre las antenas involucradas. Entre las distintas propuestas analizadas, resultan destacables las que utilizan curvas de Peano y de Hilbert para lograr una mayor inductancia el menor área posible, como se analiza en \cite{McVay:Peano}. Estos EBGs, que presentan un ancho de banda mayor, disminuyen la presión sobre el diseñador, quien puede utilizar la misma estructura aunque la frecuencia de trabajo varíe.

Podría resultar interesante, además, analizar posibles aplicaciones de las celdas complementarias a las propuestas, donde las porciones que no poseen cobre y las que sí, son intercambiadas, dado que poseen la misma periodicidad que las celdas originales. El uso de celdas complementarias resulta común en el estudio de \textit{Frequency Selective Surfaces} (FSS).

Otras técnicas para controlar la propagación de ondas de superficie han sido propuestas, entre las que destaca el uso de planos de tierra degenerados (DGS, \textit{degenerated ground planes}), que son muy similares a los EBGs. El principio de funcionamiento es el mismo, pero el hecho de que se estructuren geometrías sobre el plano de tierra modifica genera efectos sobre la radiación posterior.

Además de las ondas de superficie, que pueden ser atenuadas por los EBGs, existe también acoplamiento debido a modos cuasi-TEM que viajan por encima del sustrato, como una onda libre, y que no pueden ser bloqueados por las estructuras propuestas. Si se busca un desacoplamiento más notorio, es importante reducir el aporte de las mismas. Estudios en este sentido pueden consultarse en \cite{Asimonis:designoptimization}.


%%%%

%%%%