%%%%%%%%%%%%%%%%
%%  Capítulo 6: Conclusiones  %%
%%%%%%%%%%%%%%%%

% La tecnología microstrip posee costos bajos de fabricación y presenta comodidad para la implantación de componentes discretos, aunque es susceptible de generar modos indeseados y ondas de superficie. 
% La aparición de ondas de superficie es una consecuencia directa del uso de estructuras microstrip, que deben disponerse sobre un sustrato dieléctrico recubierto por un plano de tierra, el cual presenta una impedancia de superficie reactiva que permite esta forma de propagación de energía.
%La forma tradicional de eliminar estos comportamientos requiere de la disminución del ancho del sustrato que, en general, da lugar a una muy alta fragilidad. El uso de EBGs permitiría solucionar el problema.
% Es importante diseñar antenas teniendo en cuenta el usod e EBGs, y no agregarlo después, porque no anda bien.
% Los EBGs pueden servir para eliminar el acoplamiento, aunque requieren de un estudio previo porque ocupan espacio que puede ser importante.
% Los EBGs no son la unica forma. Se puede romper el sustrato, usar DGS, cambiar de sustrato, etc. Todos con ventajas y desventajas.
% La simulación mediante TLM es sencilla, intuitiva y da resultados estimativos que podrían considerarse buenos en algunos casos. Es importante lograr implementarlo eficientemente.
% El uso de software de simulación presenta importantes ventajas, aunque no se puede confiar ciegamente en él.
% El estudio del tema requeire de la consulta de mucha bibliografía. La información es dispersa y los conceptos son difusos. Falta un cuerpo de conocimiento sobre el tema un poco más cerrado/formado/formalizado. La bibliografía es poco confiable. Muchos errores, incluso errores conceptuales.
% Se esperaban mejores resultados. No puedo estar completamente seguro de por qué no funcionó. Tengo hipótesis (decirlas).
% El efecto del acoplamiento por ondas de superficie no es el único, y es importante saberlo a la hora de analizarlo.
% Las herramientas de la óptica y la fotónica resultan muy útiles para el análisis de EBGs. Realmente se puede volver interdisciplinario.
% Escribir una tesis no es sencillo.

\section{Propuestas de trabajos futuros}

El estudio de este tipo de estructuras permite la profundización sobre diversas aristas que no fueron en foco de este trabajo.

Además de las geometrías analizadas, en los últimos años han surgido diversas estructuras que permiten efectos multibanda, a partir de lograr una controlada variación geometría de las celdas unitarias, como se analiza en \cite{Kern:multiband}. Por otro lado, resulta importante lograr efectos similares utilizando celdas unitarias más pequeñas, de modo que se pueda ubicar una mayor cantidad de celdas entre las antenas \textit{microstrip} analizadas. Entre las distintas propuestas analizadas, resultan destacables las que utilizan curvas de Peano y de Hilbert para lograr una mayor inductancia el menor área posible, como se analiza en \cite{McVay:Hilbert}.

Podría resultar interesante, además, analizar posibles aplicaciones de las celdas complementarias a las propuestas, donde las porciones que no poseen cobre y las porciones que sí lo presentan son intercambiadas, dado que poseen la misma periodicidad que las celdas originales. Estudios en este sentido se observan en BUSCAR.

Otras técnicas para controlar la propagación de ondas de superficie han sido propuestas, entre las que destaca el uso de planos de tierra degenerados (DGS, \textit{degenerated ground planes}), que son muy similares a los EBGs.

Además de las ondas de superficie, que pueden ser atenuadas por los EBGs, existe también acoplamiento debido a modos cuasi-TEM que viajan por encima del sustrato, como una onda libre, y que no pueden ser bloqueados por las estructuras propuestas. Si se busca un desacoplamiento más notorio, es importante reducir el aporte de las mismas. Estudios en este sentido pueden consultarse en \cite{Asimonis:designoptimization}.
%- Efecto de leaky waves
%- Multiband EBGs, variando la geometría (kern, werner, monorchio, "the design synthesis of multiband artificial magnetic..")
%- Formas de reducir el acoplamiento de modos cuasitem que viajan por encima del sustrato y acoplan igual a las antenas. (Assimonis, Yioultsis, Antonopoulos)
%- DGS (ver el paper cheto)

%%%%

%%%%