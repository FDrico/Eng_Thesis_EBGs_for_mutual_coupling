%%%%%%%%%%%%%%%%
%%  Capítulo 6: Conclusiones  %%
%%%%%%%%%%%%%%%%


%%%%
\begin{itemize}
\item En este trabajo se han estudiado distintos tipos de alimentadores para antenas parabólicas: aberturas rectangulares y circulares, guías de onda rectangulares y cilíndricas, y bocinas rectangulares y cónicas, empleando modelos teóricos y simulaciones computacionales. Para el diseño del alimentador se contó con un reflector parabólico de diámetro 1,8 metros y una relación $F/D$ = 0,353, el cual fue construido previamente por el autor de la presente tesis. El material empleado fue tejido de alambre de aluminio de aproximadamente 1 mm de paso, y la estructura fue realizada con varillas y perfiles de aluminio.
\item A partir de las dimensiones del reflector parabólico construido, se procedió al diseño del alimentador. Un alimentador ideal es aquel que produce una iluminación uniforme sobre la superficie del reflector y además que toda la radiación sea captada por el reflector, logrando así una eficiencia de iluminación del 100 \%. Sin embargo, en la práctica los diagramas de radiación distan del caso ideal, y siempre existen pérdidas por iluminación no uniforme y por radiación que se va por fuera de los límites del reflector. Con el fin de obtener la máxima eficiencia de iluminación, el problema del diseño del alimentador se reduce a una solución de compromiso, la cual establece que la diferencia entre las potencias incidentes en el borde y en el vértice del reflector sea aproximadamente -11 dB.
\item Para hallar los campos radiados por las antenas de abertura estudiadas, se empleó un método basado en la óptica geométrica, que plantea las fuentes equivalentes sobre la abertura y se las integra, asumiendo que los campos eléctrico $\mathbf{E}_a$ y magnético $\mathbf{H}_a$ sobre la abertura constituyen una onda plana.
\item A partir del modelo teórico explicado anteriormente, se obtuvieron los diagramas de radiación de las siguientes antenas: aberturas rectangulares y circulares, guías de onda rectangulares y cilíndricas, y bocinas rectangulares y cónicas. Se estudiaron dichos diagramas de radiación variando las dimensiones para analizar cual es la antena que mejor responde a los parámetros de iluminación requeridos. Como resultado del estudio entre el modelo teórico y la simulación numérica, se obtuvo que la antena que mejor cumple con los requisitos establecidos es la guía de onda cilíndrica de 83 mm de diámetro. A diferencia de las bocinas, posee el centro de fase en el centro de la abertura, por lo que luego de reflejarse en la superficie metálica del reflector, la fase de las ondas electromagnéticas sobre el plano de abertura es constante. A la vez, debido a la geometría circular, genera una distribución espacial de campos con mayor simetría rotacional que las guías de onda rectangulares.
\item Sobre la base de la guía de onda cilíndrica elegida, se agregó a la misma un choque, que consiste en una estructura anular ubicada en las proximidades de la abertura, que presenta una reactancia idealmente infinita a las ondas de superficie. Con el agregado del choque, se logró atenuar la radiación posterior (back-scattering) en 7,5 dB y además mejorar la forma del lóbulo principal.
\item La construcción del alimentador se realizó a partir de un cilindro de latón que se mecanizó para llegar a las dimensiones que fueron calculadas teóricamente y mediante simulación numérica. La adaptación del alimentador se realizó ajustando la distancia entre el excitador y el cortocircuito de la guía de onda para minimizar la relación de onda estacionaria (ROE).
\item La pérdidas óhmicas producidas en el alimentador son insignificantes, aún cuando la frecuencia de operación se encuentre cercana a la frecuencia de corte de la guía de onda y la conductividad del material conductor no sea tan elevada como la del cobre y la del aluminio, por lo que puede considerarse que la ganancia es prácticamente igual a la directividad.
\item Los resultados obtenidos del alimentador con el agregado del choque fueron:
\begin{itemize}
\item Diámetro de la guía de onda: 79,2 mm
\item Ganancia: 7,41 dBi
\item ROE: 1,06
\item Relación frente-espalda: 17,71 dB
\item Relación de polarización cruzada: 18,63 dB
\item Ancho de haz principal (Plano E): 82,37$^{\circ}$
\item Ancho de haz principal (Plano H): 85,22$^{\circ}$
\item Ganancia normalizada en el ángulo $\theta_0$ (Plano E): -7,01 dB
\item Ganancia normalizada en el ángulo $\theta_0$ (Plano H): -8,60 dB
\end{itemize}
\item Los resultados obtenidos del conjunto formado por el reflector parabólico y el alimentador fueron:
\begin{itemize}
\item Ganancia: 31,68 dBi
\item ROE: 1,06
\item Eficiencia de abertura: 71,93 \%
\item Relación de polarización cruzada: 30,85 dB
\item Ancho de haz principal (Plano E): 4,68$^{\circ}$
\item Ancho de haz principal (Plano H): 4,68$^{\circ}$
\item Ángulo del primer lóbulo secundario (Plano E): 7,17$^{\circ}$
\item Ángulo del primer lóbulo secundario (Plano H): 7,45$^{\circ}$
\item Amplitud normalizada del primer lóbulo secundario (Plano E): -25,77 dB
\item Amplitud normalizada del primer lóbulo secundario (Plano H): -25,87 dB
\end{itemize}
\end{itemize}
%%%%