%%%%%%%%%%%%%%%%
%%  Capítulo 6: Conclusiones  %%
%%%%%%%%%%%%%%%%

% Es importante diseñar antenas teniendo en cuenta el usod e EBGs, y no agregarlo después, porque no anda bien.
% Los EBGs pueden servir para eliminar el acoplamiento, aunque requieren de un estudio previo porque ocupan espacio que puede ser importante.
% Los EBGs no son la unica forma. Se puede romper el sustrato, usar DGS, cambiar de sustrato, etc. Todos con ventajas y desventajas.
% La simulación mediante TLM es sencilla, intuitiva y da resultados estimativos que podrían considerarse buenos en algunos casos. Es importante lograr implementarlo eficientemente.
% El uso de softwar de simulación presenta importantes ventajas, aunque no se puede confiar ciegamente en él.
% El estudio del tema requeire de la consulta de mucha bibliografía. La información es dispersa y los conceptos son difusos. Falta un cuerpo de conocimiento sobre el tema un poco más cerrado/formado/formalizado. La bibliografía es poco confiable. Muchos errores, incluso errores conceptuales.
% Se esperaban mejores resultados. No puedo estar completamente seguro de por qué no funcionó. Tengo hipótesis (decirlas).
% El efecto del acoplamiento por ondas de superficie no es el único, y es importante saberlo a la hora de analizarlo.
% Las herramientas de la óptica y la fotónica resultan muy útiles para el análisis de EBGs. Realmente se puede volver interdisciplinario.
% Escribir una tesis no es sencillo.

% Trabajos a futuro
%-Analizar cómo cambia una estructura cuando se usan estructuras complementarias. Varia comportamiento? cambia BG? Es importante tenerlo en cuenta?
%-Miniaturización. Celdas más pequeñas. Curvas de Hilbert (McVay, Engheta) y Peano.
%- Efecto de leaky waves
%- Multiband EBGs, variando la geometría (kern, werner, monorchio, "the design synthesis of multiband artificial magnetic..")
%- Formas de reducir el acoplamiento de modos cuasitem que viajan por encima del sustrato y acoplan igual a las antenas. (Assimonis, Yioultsis, Antonopoulos)
%- DGS (ver el paper cheto)

%%%%

%%%%