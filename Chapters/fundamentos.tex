%%%%%%%%%%%%%%%%%%%%%%%%%%
%%  Capítulo 2: Fundamentos de estructuras de banda prohibida electromagnética (EBG)}  %%
%%%%%%%%%%%%%%%%%%%%%%%%%%

%%%%
\section{Reseña histórica: Metamateriales, materiales periódicos y EBGs}
\label{sec_reseña_metamateriales}
%%%%
% EBGs in Antennas, Rahmat (libro). Pagina 2.
% Explicación de los métodos de análisis
% Uso para low profile antennas
% Engheta, pag 215
% Caloz, ^pag 17
% Caloz Pag 20
% Caloz pag 83, diferencia con filtros
% Pozar, pag 380
% Pozar, p 424. Stepped impedance filters.
% Pozar, bandstop filter design, pag 440
% Tabla interesante: Baccarello, Paulotto, impreso.
% stack de capas o cristal? Brown, McMahon, Parker. Ademas linda foto.
% Relación con soft-hard surfaces. Gao, chen, wang, yang-
% Discusión terminológica: Oliner.
% ITOH, Pediodic structures.
% DGS, tesis de ABIDIN, pag2. Tiene intro histórica.
% ABIDIN, tesis, pagina 16: Def de EBG. imagen interesante.
% Importancia de EBGs. Venkateswaran tesis, pag 1.
\section{Difracción de Bragg}
\label{sec_bragg}
%%%%
% Caloz, pag 19
% Kamgaing (tesis), pagina22
\section{Bloch-Floquet}
\label{sec_bloch}
%%%%
% Libro de rahmat. Pagina 46.
% Zona de brillouin. Rahmat, libro, pagina 31.
% Diagrama de dispersión. Página 67 Rahmat. Caloz: pagina 106.
% Impedancia de Bloch: Caloz, pagina 113
% Tesis Choi, pag 82.
% Tesis Kovacs, pag 13.
\section{Impedancia de onda y de superficie}
\label{sec_imp_superficie}
%%%%
% Rahmat. Pagina 77 libro.
% Engheta, pagina 290.



\section{Metamateriales ópticos: Cristales fotónicos}
\label{sec_cristales_fotonicos}
%%%%
% Lightline stuff. Rahmat, pagina 28. Caloz, pagina 139. Pozar pag 386
\section{Tipos de EBG}
\label{sec_tipos_mtm}
%%%%
% Tesis de kovacs, pagina 7.
\subsection{EBGs de mano izquierda}
\label{subsec_ebg_izquierda}
%%%%
% Caloz, Pag 27, 43, 
\subsection{EBGs uniplanares}
\label{subsec_ebg_uniplanar}
%%%%
% buena intro en rahmat (libro), pagina 35-37
% Buscar las distintas formas, incluyendo Peano. Relación con FSS.
% Buena intro esn Goussetis, Feresidis, Vardaxoglou.
% Diseños para incidencia obliucua: Kim, Yand, Elsherbeni. Tambien en Lin, Li, Zhang.
% Hilbert: McVAy, Engheta, Hoorfar.
% Power loss analysis: Mohajer-Iravani, Ramahi, de Hindawi corp.
% Kern, Douglas, Werner.
% Analisis en Goussetis, Feresidis, paper posta. Tipos de resonancia.
% Lamminen, Vimpari, Saily. 
% Maci, Caiazzo, Cucini. Jodido, interesante. Leer.
\section{Modelado y simulación de metamateriales}
\label{sec_simulacion_mtm}
%%%%
% Un modelado medio simple se puede ver en la tesis de Choi, pag 90-99
\subsection{Métodos de cavidades periódicas}
\label{subsec_eigenfunctions}
%%%%
% Fullwave: Baccarello, Paulotto, impreso.
% Hacer estudio gráfico similar al CALOZ, pag 176.
\subsection{Modelado por líneas de transmisión}
%%%%
% Caloz, pag 60, 67, 76, 79
% Bidimensional, Caloz, pag 133
% Rahman Stuchly.ç
% Calculo de la inductancia de meander inductors: Stojanovic, Zivanov
% Cuentas! Wu, Lin, Wang, Wang, Chen
% Kim, Schutt-Ainé, 178. Modelado para PDN.
\subsubsection{TMM}
%%%%
% Caloz, pagina 144
% Choi (tesis), pag 54
\subsubsection{TLM}
\label{subsubsec_tlm}
%%%%
% Caloz, pag 155