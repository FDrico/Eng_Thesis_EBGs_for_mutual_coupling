%%%%%%%%%%%%%%%%%%%%%%%%%%
%%  Capítulo 2: Fundamentos de estructuras de banda prohibida electromagnética (EBG)}  %%
%%%%%%%%%%%%%%%%%%%%%%%%%%

%%%%
\section{Reseña histórica: Metamateriales, materiales periódicos y EBGs}
\label{sec_resenia_metamateriales}
%%%%
% EBGs in Antennas, Rahmat (libro). Pagina 2.
% Explicación de los métodos de análisis
% Uso para low profile antennas
% Engheta, pag 215
% Caloz, ^pag 17
% Caloz Pag 20
% Caloz pag 83, diferencia con filtros
% Pozar, pag 380
% Pozar, p 424. Stepped impedance filters.
% Pozar, bandstop filter design, pag 440
% Tabla interesante: Baccarello, Paulotto, impreso.
% stack de capas o cristal? Brown, McMahon, Parker. Ademas linda foto.
% Relación con soft-hard surfaces. Gao, chen, wang, yang-
% Discusión terminológica: Oliner.
% ITOH, Pediodic structures.
% DGS, tesis de ABIDIN, pag2. Tiene intro histórica.
% ABIDIN, tesis, pagina 16: Def de EBG. imagen interesante.
% Venkateswaran, algunos ejemplos de EBGs.
% Importancia de EBGs. Venkateswaran tesis, pag 1.
% Algo de intro: Pagina 4, tesis Zheng.

Los EBG (Materiales de banda prohibida electromagnética, \textit{Electromagnetic Bandgap Materials}), PBG (Materiales de banda prohibida fotónica, \textit{Photonic Bandgap Materials}) o cristales fotónicos son un tipo de estructura artificial, dieléctrica o metalodieléctrica, con capacidades para controlar ondas electromagnéticas \cite{Engheta} a partir de una variación periódica en el espacio de la constante dieléctrica, permitiendo que las mismas se propaguen sólo en direcciones determinadas, o impedir su propagación completamente. Esta capacidad proviene de su estructura de banda prohibida electromagnética, que surge en analogía al concepto de banda prohibida electrónica, que controla el movimiento de ondas de electrones que viajan en un potencial periódico cristalino.

En 1946, Louis Brillouin publicó su libro sobre ondas mecánicas, "\textit{Wave propagation in periodic structures: Electric filters and crystal lattices}" \cite{Brillouin:WavePropagation}, donde demostró que un arreglo periódico impone restricciones a los vectores de onda $k$ que pueden propagarse por él, dado que da lugar a condiciones de contorno para los modos permitidos. Aquellas ondas que no cumplieran las condiciones de borde impuestas, no podrían propagarse. La propagación de ondas sobre superficies periódicas había comenzado a estudiarse, para ondas electromagnéticas, con anterioridad al trabajo de Brillouin, aunque a partir del mismo se formalizó el desarrollo de las aplicaciones para microondas.

En 1968, el físico ruso Viktor Veselago describió por primera vez, de forma teórica, la posibilidad de que existieran materiales con índice de refracción negativo, denominados LHS (\textit{Left Handed Materials}, materiales de mano izquierda), con permitividad eléctrica y permeabilidad magnética negativas, de modo que la velocidad de fase de una onda que se propagara por ese medio fuera antiparalela a la velocidad de grupo. Treinta años más tarde, Pendry propuso la primera forma de fabricación de un medio con esas características en una banda limitada de frecuencia, utilizando SSRs (Split-ring resonators, resonadores de aro dividido), responsables de la permeabilidad magnética negativa, y cables conductores, responsables de la permitividad eléctrica negativa. Recién en el año 2000 se construyó el primer metamaterial sobre las propuestas de Pendry, y varios experimentos confirmaron refracción negativa en los mismos.

La primera descripción de estructuras dieléctricas periódicas con una banda prohibida completa fue dada, en 1990, por Ho, Chan y Soukoulis, en Iowa, Estados Unidos, quienes propusieron un arreglo periódico de esferas dieléctricas, dispuestas en forma de capas. Para un rango amplio de radios de esferas, el efecto de banda prohibida se daba en todas direcciones. Hacia fines de la década de 1980, Yablonovitch divisó una estructura cristalina simétrica de más fácil fabricación, que consistía en la producción de tres agujeros cilíndricos en un prisma dieléctrico, repetido periódicamente. Demostró, además, usando dichas estructuras, que la existencia de una banda prohibida electromagnética podía ser predicha teóricamente, en base, principalmente, a la constante de periodicidad del dieléctrico artificial.

Los trabajos de Yablonovitch y Pendry se hicieron sobre estructuras con banda de interés en microondas, pero bajo modelos fotónicos, gracias a la linealidad de las ecuaciones de Maxwell, lo que permitió el uso de distintas técnicas conocidas y desarrolladas durante el siglo XX en microondas para el diseño de materiales fotónicos de estas características a partir de finales de la década de 1990, aunque principalmente en ámbitos de investigación de ciencias básicas, sin un interés marcado de la industria.

Las denominadas "superficies electromagnéticas" (\textit{electromagnetic surfaces}) consisten en superficies texturadas que imponen condiciones de contorno particulares, capaces de lograr cambiar la polarización de una onda incidente, influir sobre las ondas de superficie y controlar la fase de reflexión. La más simple consiste en una placa metálica coarrugada, como la mostrada en la figura \ref{fig:superficie-coarrugada}, de forma que las variaciones de altura sean de $\lambda/4$, que puede comportarse como una "superficie blanda" (\textit{soft surface}) o una "superficie dura" (\textit{hard surface}), en función de la polarización de la onda que se propaga.

\section{Difracción de Bragg}
\label{sec_bragg}
%%%%
% Caloz, pag 19
% Kamgaing (tesis), pagina22
\lipsum
\section{Bloch-Floquet}
\label{sec_bloch}
%%%%
\lipsum
% Libro de rahmat. Pagina 46.
% Zona de brillouin. Rahmat, libro, pagina 31.
% Diagrama de dispersión. Página 67 Rahmat. Caloz: pagina 106.
% Impedancia de Bloch: Caloz, pagina 113
% Tesis Choi, pag 82.
% Tesis Kovacs, pag 13.
% Venkateswaran, pag 26.
% Tesis Zheng, pag 5-8
\section{Impedancia de onda y de superficie}
\label{sec_imp_superficie}
%%%%
% Rahmat. Pagina 77 libro.
% Engheta, pagina 290.
\lipsum


\section{Metamateriales ópticos: Cristales fotónicos}
\label{sec_cristales_fotonicos}
%%%%
\lipsum
% Lightline stuff. Rahmat, pagina 28. Caloz, pagina 139. Pozar pag 386
\section{Tipos de EBG}
\label{sec_tipos_mtm}
%%%%
\lipsum[1]
% Tesis de kovacs, pagina 7.
\subsection{EBGs de mano izquierda}
\label{subsec_ebg_izquierda}
%%%%
\lipsum
% Caloz, Pag 27, 43, 
\subsection{EBGs uniplanares}
\label{subsec_ebg_uniplanar}
%%%%
\lipsum
% buena intro en rahmat (libro), pagina 35-37
% Buscar las distintas formas, incluyendo Peano. Relación con FSS.
% Buena intro esn Goussetis, Feresidis, Vardaxoglou.
% Diseños para incidencia obliucua: Kim, Yand, Elsherbeni. Tambien en Lin, Li, Zhang.
% Hilbert: McVAy, Engheta, Hoorfar.
% Power loss analysis: Mohajer-Iravani, Ramahi, de Hindawi corp.
% Kern, Douglas, Werner.
% Analisis en Goussetis, Feresidis, paper posta. Tipos de resonancia.
% Lamminen, Vimpari, Saily. 
% Maci, Caiazzo, Cucini. Jodido, interesante. Leer.
\section{Modelado y simulación de metamateriales}
\label{sec_simulacion_mtm}
%%%%
\lipsum
% Un modelado medio simple se puede ver en la tesis de Choi, pag 90-99
\subsection{Métodos de cavidades periódicas}
\label{subsec_eigenfunctions}
%%%%
\lipsum
% Fullwave: Baccarello, Paulotto, impreso.
% Hacer estudio gráfico similar al CALOZ, pag 176.
\subsection{Modelado por líneas de transmisión}
%%%%
\lipsum
% Caloz, pag 60, 67, 76, 79
% Bidimensional, Caloz, pag 133
% Rahman Stuchly.ç
% Calculo de la inductancia de meander inductors: Stojanovic, Zivanov
% Cuentas! Wu, Lin, Wang, Wang, Chen
% Kim, Schutt-Ainé, 178. Modelado para PDN.
%Venkateswaran, pag 40 en adelante.
% Impedancia: hacer algo similar. Tesis Zheng, pag 8

\subsubsection{TMM}
%%%%
\lipsum[2]
% Caloz, pagina 144
% Choi (tesis), pag 54
\subsubsection{TLM}
\label{subsubsec_tlm}
%%%%
\lipsum
% Caloz, pag 155