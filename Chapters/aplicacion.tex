%%%%%%%%%%%%%%%%%%%%%%%
%%  Capítulo 4: Aplicación a estructras microstrip  %%
%%%%%%%%%%%%%%%%%%%%%%%

%%%%
\section{Introducción}
% Coccioli, Yang, Itoh, intro historica.
\section{Técnicas de aumento del ancho de banda para el análisis}
\label{sec_aumento_bw}
%%%%
% Rahmat. Pagina 127, libro. Efecto de algunos parámetros.}
% Paper Kumar, Gupta, Nonradiating
% Notar que paper de Assimonis, Yioultsis no usa esto. MAL.
%%%%
\section{Diseño de la antena microstrip}
\label{sec_disenio_microstrip}
%%%%
% Pablito. Cst.
% Condiciones de borde: LIbroSinNombre, pagina 398.
%%%%
\section{Elección del metamaterial}
\label{sec_eleccion}
%%%%
% Cantidad de celdas minima: Bouali, Aguili, impreso.
% Pedir que bw del metamaterial sera mayor al de la antena
% Mirar el analisis de Coccioli, Yang, Itoh, pag 2126
% ¿criterio para la distancia? IMPORTANTE!!!!!
%%%%
\section{Estudio del efecto sobre el acoplamiento mutuo}
\label{sec_estudio_acoplam_mutuo}
%%%%
% Ondas de superficie se proparagan por plano E? Pagina 132. Rahmat, lbro,.
% Pagina 138. Rahmat.
% Efecto según la orientacion de parches. ABidin, Tesis, Pagina 37.
% Algo se puede ver en Assimonis, Yioultsis, Antonopoulos
%%%%
\section{Estudio del efecto de la distancia sobre el ancho de banda}
\label{sec_efecto_distancia}
%%%%
%Coccioli, Yang, Itoh, pag 2127
% Iluz, Shavit, pag 1448

% Ver que overall las conclusiones son similares a Islam,Alam.
