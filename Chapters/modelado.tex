%%%%%%%%%%%%%%%%%%%%%%%%%
%%  Capítulo 3: Modelado y simulacion de metamateriales  %%
%%%%%%%%%%%%%%%%%%%%%%%%%

%%%%
\section{Análisis de estructuras periódicas}
\label{sec_estructuras_periodicas}
%%%%
% Libro de rahmat. Pagina 59.
%%%%
\section{Análisis de estructuras planares propuestas}
\label{sec_estructuras_propuestas}
%%%%
% Comentar cómo se logró periodicidad, vinculando las fases.
%%%%
\section{Estudio de estructuras mediante TLM}
\label{sec_estudio_tlm}
%%%%
%Pasos: Barchloui: Simulation of FSS surfaces using 3d TLM, 2003. Primer paper de un librito. Buena explciaci{on.}
% Paper importante: Hoefer 1985. Johns, 1971.
% Leer Janyani, Paul, TLM modelling of nonlinear optical effects in fibre bragg gratings, 2004.
% Kim Kim Kang, Yook: Modeling and analysis of ebg in power distribution networks.
% sadiku

%%%%
\subsection{Algoritmo utilizando programación orientada a objetos}
\label{subsec_estudio_tlm}
%%%%