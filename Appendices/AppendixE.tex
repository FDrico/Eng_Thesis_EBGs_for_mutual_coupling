A partir de la expresión matricial del sistema de segundo orden presentado en la sección \ref{sec:bloch-floquet}, y reescrito debajo por comodidad, se puede obtener una solución utilizando el método de la exponencial matricial.

\begin{equation}
\begin{bmatrix}
\phi'_1(x) \\
\phi'_2(x)
\end{bmatrix}
=
\begin{bmatrix}
0 & 1 \\
\gamma(x)^2 & 0
\end{bmatrix}
\begin{bmatrix}
\phi_1(x) \\
\phi_2(x)
\end{bmatrix},
\end{equation}

Una solución general a cualquier sistema $\textbf{X}'(x) = A \textbf{X}(x)$ se puede expresar como

\begin{align}
	\textbf{X}(x) = e^{x\textbf{A}} \textbf{C}, \;\;\textbf{C} = [C_1 C_2 ... C_n ]^{T},
\end{align}

donde $\textbf{C}$ es un vector arbitrario que surge de la aplicación de las condiciones del problema.

De esta forma, la solución se obtiene si se puede expresar la matriz $e^{x\textbf{A}}$, que es una generalización de la serie de Maclaurin\footnote{La serie de Maclaurin es un desarrollo en serie de Taylor alrededor del 0: 

\begin{align}
	f(x) = f(0) + f'(0) x + \frac{f''(0)}{2!} x^2 + \frac{f^(3)(0)}{3!} x^3 + ... + \frac{f^(n)(0)}{n!} x^n + ...
\end{align}} para matrices.

La serie de MacLaurin establece, para el caso de una exponencial,

\begin{align}
	e^{at} = 1 + at + \frac{a^2 t^2}{2!} + \frac{a^3 t^3}{3!} + ...,
\end{align}

y para el caso de una matriz cuadrada A, donde $A^0 = \textbf{I}, \textbf{A}^1 = \textbf{A}, \textbf{A}^2 = \textbf{A} \cdot \textbf{A}$, etc:

\begin{align}
e^{x\textbf{A}} = I + x\textbf{A} + \frac{x^2 \textbf{A}^2}{2!} + \frac{x^3 \textbf{A}^3}{3!} + ...
\end{align}

Se puede demostrar que:

\begin{itemize}
	\item Si $A$ es una matriz de ceros, entonces $e^{x\textbf{A}} = e^0 = \textbf{I}$.
	\item Si $A = I$, entonces $e^{x\textbf{A}} = e^{x\textbf{I}} = e^{x} \textbf{I}$.
	\item $\frac{d}{dx} e^{x\textbf{A}} = \textbf{A} e^{x\textbf{A}}$.
	\item Si $\textbf{H}$ es una matriz de transformación lineal no singular, entonces $\textbf{A} = \textbf{H} \textbf{M} \textbf{H}^{1}$ y $e^{x\textbf{A}} = \textbf{H} e^{x\textbf{M}} \textbf{H}^{-1}$.
\end{itemize}

El mecanismo para obtener la solución del sistema utilizando estas propiedades es el siguiente:

\begin{enumerate}
	\item Encontrar los autovalores $\lambda_i$ y autovectores de $\textbf{A}$, dado que $\textbf{A}$ es un operador lineal.
	
	Dado que
	
	\begin{align}
	\textbf{A} = 
	\begin{bmatrix}
	0 & 1 \\
	\gamma(x)^2 & 0
	\end{bmatrix}
	\end{align}
	
	los autovalores son $\lambda_1 = \gamma(x)$ y $\lambda_2 = -\gamma(x)$, dos autovalores distintos.
	
	Los autovalores, en tanto, resultan $[1,\gamma(x)]$ y $[1,-\gamma(x)]$.
	
	\item Construir la matriz $\textbf{H}$ utilizando los autovectores obtenidos, dispuestos como columnas, y obtener $\textbf{H}^{-1}$.
	
	\begin{align}
	\textbf{H} = 
	\begin{bmatrix}
	1 & 1 \\
	\gamma(x) & -\gamma(x)
	\end{bmatrix},
	\end{align}
	
	por lo que
	
	\begin{align}
	\textbf{H}^{-1} = 
	\begin{bmatrix}
	1 & \frac{1}{\gamma(x)} \\
	1 & -\frac{1}{\gamma(x)}
	\end{bmatrix}.
	\end{align}	
	
	\item Obtener la forma de Jordan, $\textbf{J} = \textbf{H}^{-1} \textbf{A} \textbf{H}$.
	
	\begin{align}
	\textbf{J} = \gamma(x) 
	\begin{bmatrix}
	1 & 0 \\
	0 & -1
	\end{bmatrix}.
	\end{align}
	
	\item Componer, con la matriz $J$ obtenida, la expresión, en base a las propiedades descriptas antes, $e^{x\textbf{J}} = \textbf{H}^{-1} e^{x\textbf{A}} \textbf{H}$.
	
	\begin{align}
	e^{x\textbf{J}} =  
	\begin{bmatrix}
	e^{x \gamma(x)} & 0 \\
	0 & e^{-x \gamma(x)}
	\end{bmatrix}.
	\end{align}
	
	\item Obtener $e^{x\textbf{A}} = \textbf{H} e^{x\textbf{J}} \textbf{H}^{-1}$.
	
	\begin{align}
	e^{x\textbf{A}} =  e^{x\gamma(x)}
	\begin{bmatrix}
	1+e^{-2x \gamma(x)} & \frac{1-e^{-2x\gamma(x)}}{\gamma(x)} \\
	\gamma(x) (1-e^{-2x\gamma(x)}) & 1+e^{-2x \gamma(x)}
	\end{bmatrix}.
	\end{align}
	
\end{enumerate}

De modo que la solución final resulta

\begin{align}
\textbf{X}(x) = e^{x\textbf{A}} \textbf{C}, \;\;\textbf{C} = [C_1 C_2]^{T},
\end{align}

que depende de las condiciones del problema.