%\newlength{\longitudNtheta}
%\settowidth{\longitudNtheta}{$N_{\theta}$}
%%\the\longitudNtheta\\
%\newlength{\longitudNphi}
%\settowidth{\longitudNphi}{$N_{\phi}$}
%%\the\longitudNphi\\
%\newlength{\longitudLtheta}
%\settowidth{\longitudLtheta}{$L_{\theta}$}
%%\the\longitudLtheta\\
%\newlength{\longitudLphi}
%\settowidth{\longitudLphi}{$L_{\phi}$}
%%\the\longitudLphi\\
%\newlength{\longitudEtheta}
%\settowidth{\longitudEtheta}{$E_{\theta}$}
%%\the\longitudEtheta\\
%\newlength{\longitudEphi}
%\settowidth{\longitudEphi}{$E_{\phi}$}
%%\the\longitudEphi\\
%\newlength{\longitudEthetanor}
%\settowidth{\longitudEthetanor}{$\lvert E_{\theta}^{\circ}\left(\theta,\phi\right)\rvert^2$}
%%\the\longitudEthetanor\\
%\newlength{\longitudEphinor}
%\settowidth{\longitudEphinor}{$\lvert E_{\phi}^{\circ}\left(\theta,\phi\right)\rvert^2$}
%%\the\longitudEphinor\\

%%%%
\section{Abertura circular con plano conductor infinito y excitación con campo uniforme}
\label{subsec_apendice_c_abert_circ_inf_uni}
%%%%

%%%%
Los campos sobre la abertura están dados por:
%%%%
\begin{align}
\mathbf{E}_a &= 
\begin{cases}
\makebox[0pt][l]{$E_0\,\versor{y}$}\hphantom{-E_0\,\versor{x}} &\text{si } \rho' \leq a\\
0  &\text{caso contrario}
\end{cases}
\label{ec_apC:1}\\
\mathbf{H}_a &= 
\begin{cases} 
-E_0\,\versor{x} &\text{si } \rho' \leq a\\
0  & \text{caso contrario}
\end{cases}
\label{ec_apC:2}
\end{align}
%%%%
Empleando el modelo equivalente de la Figura \ref{fig_fundamentos:8}, las densidades de corriente quedan expresadas como:
%%%%
\begin{align}
\mathbf{M}_s &=
\begin{cases}
2E_0\,\versor{x} &\text{si } \rho' \leq a\\
0 &\text{caso contrario}
\end{cases}
\label{ec_apC:3}\\
\mathbf{J}_s &= \makebox[0pt][l]{$0$} 
\hphantom{\begin{cases} 2E_0\,\versor{x}\\0\end{cases}}\kern-\nulldelimiterspace
\forall\,x',y'
\label{ec_apC:4}
\end{align}
%%%%
Considerando que:
%%%%
\begin{subequations}
\label{grup_ec_apC:1}
\begin{align}
M_\rho  &= M_x\cos\phi '
\label{ec_apC:5}\\
M_\phi &= -M_x\sen\phi '
\label{ec_apC:6}
\end{align}
\end{subequations}
%%%%
las expresiones \eqref{grup_ec_fundamentos:6} resultan:
%%%%
\begin{subequations}
\label{grup_ec_apC:2}
\begin{align}
N_{\theta} &= 0
\label{ec_apC:7}\\
N_{\phi} &= 0
\label{ec_apC:8}\\
L_{\theta}  &= 2E_0\cos\theta\cos\phi\!\int_{0}^{\,a}\!\rho '\,d\rho '\!\int_{0}^{\,2\pi}\!e^{jk\rho '\sen\theta\cos\left(\phi - \phi '\right)}\,d\phi '
\label{ec_apC:9}\\
L_{\phi} &= -2E_0\sen\phi\!\int_{0}^{\,a}\!\rho '\,d\rho '\!\int_{0}^{\,2\pi}\!e^{jk\rho '\sen\theta\cos\left(\phi - \phi '\right)}\,d\phi '
\label{ec_apC:10}
\end{align}
\end{subequations}
%%%%
A partir de la integral definida \cite{walfram_bessel_first}:
%%%%
\begin{align}
\int_{0}^{\,2\pi}\!e^{jz\cos\left(\phi - \phi '\right)}\,d\phi ' = 2\pi J_0\!\left(z\right)
\label{ec_apC:11}
\end{align}
%%%%
las expresiones \eqref{ec_apC:9} y \eqref{ec_apC:10} se reducen a:
%%%%
\begin{subequations}
\label{grup_ec_apC:3}
\begin{align}
L_{\theta} &= 4\pi E_0\cos\theta\cos\phi\!\int_{0}^{\,a}\!\rho 'J_0\!\left(k\rho '\sen\theta\right)d\rho '
\label{ec_apC:12}\\
L_{\phi} &= - 4\pi E_0\sen\phi\!\int_{0}^{\,a}\!\rho 'J_0\!\left(k\rho '\sen\theta\right)d\rho '
\label{ec_apC:13}
\end{align}
\end{subequations}
%%%%
donde $J_0\!\left(z\right)$ es la Función de Bessel de primera especie y orden cero.

Realizando el reemplazo de variables:
%%%%
\begin{align}
t = k\rho '\sen\theta\Longrightarrow dt = k\sen\theta\, d\rho '
\label{ec_apC:14}
\end{align}
%%%%
$L_{\theta}$ y $L_{\phi}$ se expresan como:
%%%%
\begin{subequations}
\label{grup_ec_apC:4}
\begin{align}
L_{\theta} &= 4\pi E_0\frac{\cos\theta\cos\phi}{\left(k\sen\theta\right)^2}\!\int_{0}^{\,ka\sen\theta}\!tJ_0\!\left(t\right)dt
\label{ec_apC:15}\\
L_{\phi} &= - 4\pi E_0\frac{\sen\phi}{\left(k\sen\theta\right)^2}\!\int_{0}^{\,ka\sen\theta}\!tJ_0\!\left(t\right)dt
\label{ec_apC:16}
\end{align}
\end{subequations}
%%%%
Utilizando la propiedad \cite{walfram_bessel_first}:
%%%%
\begin{align}
\int_{0}^{\,\beta}\!zJ_0\!\left(z\right)dz = \beta J_1\!\left(\beta\right)
\label{ec_apC:17}
\end{align}
%%%%
las expresiones \eqref{grup_ec_apC:4} se reducen a:
%%%%
\begin{subequations}
\label{grup_ec_apC:5}
\begin{align}
L_{\theta} &= 4\pi a^2E_0\!\left[\cos\theta\cos\phi\left(\frac{J_1\!\left(ka\sen\theta\right)}{ka\sen\theta}\right)\right]
\label{ec_apC:18}\\
L_{\phi} &= - 4\pi a^2E_0\!\left[\sen\phi\left(\frac{J_1\!\left(ka\sen\theta\right)}{ka\sen\theta}\right)\right]
\label{ec_apC:19}
\end{align}
\end{subequations}
%%%%
donde $J_1\!\left(z\right)$ es la Función de Bessel de primera especie y orden uno.

Introduciendo las expresiones \eqref{grup_ec_apC:5}, los campos radiados por la abertura \eqref{grup_ec_fundamentos:3} pueden expresarse como:
%%%%
\begin{subequations}
\label{grup_ec_apC:6}
\begin{align}
E_r &\simeq 0
\label{ec_apC:20}\\
E_\theta &\simeq j\frac{a^2kE_0e^{-jkr}}{r}\left[\sen\phi\left(\frac{J_1\!\left(ka\sen\theta\right)}{ka\sen\theta}\right)\right]
\label{ec_apC:21}\\
E_\phi &\simeq j\frac{a^2kE_0e^{-jkr}}{r}\left[\cos\theta\cos\phi\left(\frac{J_1\!\left(ka\sen\theta\right)}{ka\sen\theta}\right)\right]
\label{ec_apC:22}\\
H_r &\simeq 0
\label{ec_apC:23}\\
H_\theta &\simeq -\frac{E_\phi}{\eta}
\label{ec_apC:24}\\
H_\phi &\simeq +\frac{E_\theta}{\eta}
\label{ec_apC:25}
\end{align}
\end{subequations}
%%%%
Para hallar la intensidad de radiación a partir de la expresión \eqref{ec_intro:15}, primero se determina $\lvert E_{\theta}^{\circ}\left(\theta,\phi\right)\rvert^2$ y $\lvert E_{\phi}^{\circ}\left(\theta,\phi\right)\rvert^2$, cuyas expresiones son:
%%%%
\begin{subequations}
\label{grup_ec_apC:7}
\begin{align}
\lvert E_{\theta}^{\circ}\left(\theta,\phi\right)\rvert^2 &= \frac{a^4k^2\left|E_0\right|^2}{r^2}\sen^2\phi\left(\frac{J_1\!\left(ka\sen\theta\right)}{ka\sen\theta}\right)^2
\label{ec_apC:26}\\
\lvert E_{\phi}^{\circ}\left(\theta,\phi\right)\rvert^2 &= \frac{a^4k^2\left|E_0\right|^2}{r^2}\cos^2\theta\cos^2\phi\left(\frac{J_1\!\left(ka\sen\theta\right)}{ka\sen\theta}\right)^2
\label{ec_apC:27}
\end{align}
\end{subequations}
%%%%
La expresión de la intensidad de radiación resulta:
%%%%
\begin{align}
U = \frac{a^4k^2\left|E_0\right|^2}{2\eta}\left(\sen^2\phi + \cos^2\theta\cos^2\phi \right)\!\left(\frac{J_1\!\left(ka\sen\theta\right)}{ka\sen\theta}\right)^2
\label{ec_apC:28}
\end{align}
%%%%
De la expresión \eqref{ec_apC:28} se obtiene el factor de diagrama de potencia $F\left(\theta,\phi\right)$ y la constante $B_0$, por lo que:
%%%%
\begin{gather}
F\left(\theta,\phi\right) = \left(\sen^2\phi + \cos^2\theta\cos^2\phi \right)\!\left(\frac{J_1\!\left(ka\sen\theta\right)}{ka\sen\theta}\right)^2
\label{ec_apC:29}\\
B_0 = \frac{a^4k^2\left|E_0\right|^2}{2\eta}
\label{ec_apC:30}
\end{gather}
%%%%
La directividad se calcula a partir de la expresión \eqref{ec_intro:18}, empleando el factor de diagrama de potencia determinado en la expresión \eqref{ec_apC:29}.
%%%%

%%%%
\section{Abertura circular con plano conductor infinito y excitación con campo sinusoidal en modo dominante}
\label{subsec_apendice_c_abert_circ_inf_dom}
%%%%

%%%%
A partir de las componentes transversales de los campos en una guía de onda cilíndrica para modo dominante, cuyas expresiones son \eqref{grup_ec_apA:29}, se deduce que los campos sobre la abertura están dados por:
%%%%
\begin{align}
\mathbf{E}_a &= 
\begin{cases} 
\makebox[0pt][l]{$E_0\,\dfrac{\sen\phi '}{\rho '}\,J_1\!\left(\dfrac{\chi '_{11}}{a}\rho '\right)\!\versor{\uprho} + E_0\cos\phi '\,{J_1}'\!\left(\dfrac{\chi '_{11}}{a}\rho '\right)\!\versor{\upphi}$}\hphantom{- \dfrac{E_0}{\eta}\cos\phi '\,{J_1}'\!\left(\dfrac{\chi '_{11}}{a}\rho '\right)\!\versor{\uprho} + \dfrac{E_0}{\eta}\dfrac{\sen\phi '}{\rho '}\,J_1\!\left(\dfrac{\chi '_{11}}{a}\rho '\right)\!\versor{\upphi}} &\text{si } \rho '\leq a\\
0 &\text{caso contrario}
\end{cases}
\label{ec_apC:31}\\
\mathbf{H}_a &= 
\begin{cases} 
- \dfrac{E_0}{\eta}\cos\phi '\,{J_1}'\!\left(\dfrac{\chi '_{11}}{a}\rho '\right)\!\versor{\uprho} + \dfrac{E_0}{\eta}\dfrac{\sen\phi '}{\rho '}\,J_1\!\left(\dfrac{\chi '_{11}}{a}\rho '\right)\!\versor{\upphi} &\text{si } \rho '\leq a\\
0  &\text{caso contrario}
\end{cases}
\label{ec_apC:32}
\end{align}
%%%%
Empleando el modelo equivalente de la Figura \ref{fig_fundamentos:9}, las densidades de corriente quedan expresadas como:
%%%%
\begin{align}
\mathbf{M}_s &=
\begin{cases}
2E_0\cos\phi '\,{J_1}'\!\left(\dfrac{\chi '_{11}}{a}\rho '\right)\!\versor{\uprho} - 2E_0\,\dfrac{\sen\phi '}{\rho '}\,J_1\!\left(\dfrac{\chi '_{11}}{a}\rho '\right)\!\versor{\upphi} &\text{si } \rho ' \leq a\\
0 &\text{caso contrario}
\end{cases}
\label{ec_apC:33}\\
\mathbf{J}_s &= \makebox[0pt][l]{$0$} 
\hphantom{\begin{cases}2E_0\cos\phi '\,{J_1}'\!\left(\dfrac{\chi '_{11}}{a}\rho '\right)\!\versor{\uprho} - 2E_0\,\dfrac{\sen\phi '}{\rho '}\,J_1\!\left(\dfrac{\chi '_{11}}{a}\rho '\right)\!\versor{\upphi}\\0\end{cases}}\kern-\nulldelimiterspace
\forall\,x',y'
\label{ec_apC:34}
\end{align}
%%%%
y las expresiones \eqref{grup_ec_fundamentos:6} como:
%%%%
\begin{subequations}
\label{grup_ec_apC:8}
\begin{align}
N_{\theta} &= 0
\label{ec_apC:35}\\
N_{\phi} &= 0
\label{ec_apC:36}\\
L_{\theta}  &= 2E_0\cos\theta\left\{\int_{0}^{\,a}\!\rho '{J_1}'\!\left(\frac{\chi '_{11}}{a}\rho '\right)\!\left[\int_{0}^{\,2\pi}\!\cos\phi '\cos\left(\phi - \phi '\right)e^{jk\rho '\sen\theta\cos\left(\phi - \phi '\right)}\,d\phi '\right]\!d\rho '\right.\notag\\
&\left.-\int_{0}^{\,a}J_1\!\left(\frac{\chi '_{11}}{a}\rho '\right)\!\left[\int_{0}^{\,2\pi}\!\sen\phi '\sen\left(\phi - \phi '\right)e^{jk\rho '\sen\theta\cos\left(\phi - \phi '\right)}\,d\phi '\right]\!d\rho '\right\}
\label{ec_apC:37}\\
L_{\phi} &= - 2E_0\left\{\int_{0}^{\,a}\!\rho '{J_1}'\!\left(\frac{\chi '_{11}}{a}\rho '\right)\!\left[\int_{0}^{\,2\pi}\!\cos\phi '\sen\left(\phi - \phi '\right)e^{jk\rho '\sen\theta\cos\left(\phi - \phi '\right)}\,d\phi '\right]\!d\rho '\right.\notag\\
&\left.+\int_{0}^{\,a}J_1\!\left(\frac{\chi '_{11}}{a}\rho '\right)\!\left[\int_{0}^{\,2\pi}\!\sen\phi '\cos\left(\phi - \phi '\right)e^{jk\rho '\sen\theta\cos\left(\phi - \phi '\right)}\,d\phi '\right]\!d\rho '\right\}
\label{ec_apC:38}
\end{align}
\end{subequations}
%%%%
A partir de las integrales definidas \cite{walfram_bessel_first}:
%%%%
\begin{subequations}
\label{grup_ec_apC:9}
\begin{align}
\int_{0}^{\,2\pi}\!\cos\phi '\cos\left(\phi - \phi '\right)e^{jz\cos\left(\phi - \phi '\right)}\,d\phi ' &= \pi\cos\phi\left[J_0\!\left(z\right) - J_2\!\left(z\right)\right]
\label{ec_apC:39}\\
\int_{0}^{\,2\pi}\!\sen\phi '\sen\left(\phi - \phi '\right)e^{jz\cos\left(\phi - \phi '\right)}\,d\phi ' &= -\pi\cos\phi\left[J_0\!\left(z\right) + J_2\!\left(z\right)\right]
\label{ec_apC:40}\\
\int_{0}^{\,2\pi}\!\cos\phi '\sen\left(\phi - \phi '\right)e^{jz\cos\left(\phi - \phi '\right)}\,d\phi ' &= \pi\sen\phi\left[J_0\!\left(z\right) + J_2\!\left(z\right)\right]
\label{ec_apC:41}\\
\int_{0}^{\,2\pi}\!\sen\phi '\cos\left(\phi - \phi '\right)e^{jz\cos\left(\phi - \phi '\right)}\,d\phi ' &= \pi\sen\phi\left[J_0\!\left(z\right) - J_2\!\left(z\right)\right]
\label{ec_apC:42}
\end{align}
\end{subequations}
%%%%
las expresiones \eqref{ec_apC:37} y \eqref{ec_apC:38} se reducen a:
%%%%
\begin{subequations}
\label{grup_ec_apC:10}
\begin{align}
\begin{split}
L_{\theta}  &= 2\pi E_0\cos\theta\cos\phi\left\{\int_{0}^{\,a}\!\rho '{J_1}'\!\left(\frac{\chi '_{11}}{a}\rho '\right)\!\left[J_0\!\left(k\rho '\sen\theta\right) - J_2\!\left(k\rho '\sen\theta\right)\right]\!d\rho '\right.\\
&\left.+\int_{0}^{\,a}\!J_1\!\left(\frac{\chi '_{11}}{a}\rho '\right)\!\left[J_0\!\left(k\rho '\sen\theta\right) + J_2\!\left(k\rho '\sen\theta\right)\right]d\rho '\right\}
\end{split}
\label{ec_apC:43}\\
\begin{split}
L_{\phi} &= - 2\pi E_0\sen\phi\left\{\int_{0}^{\,a}\!\rho '{J_1}'\!\left(\frac{\chi '_{11}}{a}\rho '\right)\!\left[J_0\!\left(k\rho '\sen\theta\right) + J_2\!\left(k\rho '\sen\theta\right)\right]\!d\rho '\right.\\
&\left.+\int_{0}^{\,a}\!J_1\!\left(\frac{\chi '_{11}}{a}\rho '\right)\!\left[J_0\!\left(k\rho '\sen\theta\right) - J_2\!\left(k\rho '\sen\theta\right)\right]d\rho '\right\}
\end{split}
\label{ec_apC:44}
\end{align}
\end{subequations}
%%%%
Empleando las relaciones de recurrencia de las funciones de Bessel \cite{walfram_bessel_first}:
%%%%
\begin{align}
\frac{n}{z}J_n\!\left(z\right) &= \frac{1}{2}\left[J_{n - 1}\!\left(z\right) + J_{n + 1}\!\left(z\right)\right]
\label{ec_apC:45}\\
{J_n}'\!\left(z\right) &= \frac{1}{2}\left[J_{n - 1}\!\left(z\right) - J_{n + 1}\!\left(z\right)\right]
\label{ec_apC:46}
\end{align}
%%%%
se llega a las expresiones:
%%%%
\begin{align}
{J_1}'\!\left(z\right) + \frac{J_1\!\left(z\right)}{z} &= J_0\!\left(z\right)
\label{ec_apC:47}\\
{J_1}'\!\left(z\right) - \frac{J_1\!\left(z\right)}{z} &= -J_2\!\left(z\right)
\label{ec_apC:48}
\end{align}
%%%%
por lo que $L_{\theta}$ y $L_{\phi}$ se expresan como:
%%%%
\begin{subequations}
\label{grup_ec_apC:11}
\begin{align}
\begin{split}
L_{\theta} &= 2\pi\frac{\chi '_{11}}{a}E_0\cos\theta\cos\phi\left[\int_{0}^{\,a}\!\rho 'J_0\!\left(\frac{\chi '_{11}}{a}\rho '\right)\!J_0\!\left(k\rho '\sen\theta\right)d\rho '\right.\\
&+\left.\int_{0}^{\,a}\!\rho 'J_2\!\left(\frac{\chi '_{11}}{a}\rho '\right)\!J_2\!\left(k\rho '\sen\theta\right)d\rho '\right]
\end{split}
\label{ec_apC:49}\\
\begin{split}
L_{\phi} &= - 2\pi\frac{\chi '_{11}}{a}E_0\sen\phi\left[\int_{0}^{\,a}\!\rho 'J_0\!\left(\frac{\chi '_{11}}{a}\rho '\right)\!J_0\!\left(k\rho '\sen\theta\right)d\rho '\right.\\
&-\left.\int_{0}^{\,a}\!\rho 'J_2\!\left(\frac{\chi '_{11}}{a}\rho '\right)\!J_2\!\left(k\rho '\sen\theta\right)d\rho '\right]
\end{split}
\label{ec_apC:50}
\end{align}
\end{subequations}
%%%%
Utilizando la fórmula integral de Lommel \cite{walfram_bessel_first}:
%%%%
\begin{align}
&\int_{0}^{\,z}\!zJ_n\!\left(\alpha z\right)\!J_n\!\left(\beta z\right)dz = \frac{z}{\alpha^2 - \beta^2}\left[J_n\!\left(\alpha z\right)\!{J_n}'\!\left(\beta z\right) - {J_n}'\!\left(\alpha z\right)\!J_n\!\left(\beta z\right)\right]
\label{ec_apC:51}
\end{align}
%%%%
es posible definir las integrales:
%%%%
\begin{subequations}
\label{grup_ec_apC:12}
\begin{align}
\int_{0}^{\,a}\!\rho 'J_0\!\left(\alpha\rho '\right)\!J_0\!\left(\beta\rho '\right)d\rho ' &= \frac{a}{\alpha^2 - \beta^2}\left[\alpha J_0\!\left(\beta a\right)\!J_1\!\left(\alpha a\right) - \beta J_0\!\left(\alpha a\right)\!J_1\!\left(\beta a\right)\right]
\label{ec_apC:52}\\
\int_{0}^{\,a}\!\rho 'J_2\!\left(\alpha\rho '\right)\!J_2\!\left(\beta\rho '\right)d\rho ' &= \frac{a}{\alpha^2 - \beta^2}\left[\beta J_1\!\left(\beta a\right)\!J_2\!\left(\alpha a\right) - \alpha J_1\!\left(\alpha a\right)\!J_2\!\left(\beta a\right)\right]
\label{ec_apC:53}
\end{align}
\end{subequations}
%%%%
Sumando y restando las integrales \eqref{grup_ec_apC:12}, se obtiene:
%%%%
\begin{subequations}
\label{grup_ec_apC:13}
\begin{align}
\int_{0}^{\,a}\!\rho '\!\left[J_0\!\left(\alpha\rho '\right)\!J_0\!\left(\beta\rho '\right) + J_2\!\left(\alpha\rho '\right)\!J_2\!\left(\beta\rho '\right)\right]d\rho ' &= \frac{2\alpha a}{\alpha^2 - \beta^2}J_1\!\left(\alpha a\right)\!{J_1}'\!\left(\beta a\right)
\label{ec_apC:54}\\
\int_{0}^{\,a}\!\rho '\!\left[J_0\!\left(\alpha\rho '\right)\!J_0\!\left(\beta\rho '\right) - J_2\!\left(\alpha\rho '\right)\!J_2\!\left(\beta\rho '\right)\right]d\rho ' &= \frac{2}{\alpha \beta}J_1\!\left(\alpha a\right)\!J_1\!\left(\beta a\right)
\label{ec_apC:55}
\end{align}
\end{subequations}
%%%%
y a partir de las expresiones \eqref{grup_ec_apC:13}, las expresiones \eqref{grup_ec_apC:11} se reducen a:
%%%%
\begin{subequations}
\label{grup_ec_apC:14}
\begin{align}
&\hspace{\longitudLphi}\hspace{-\longitudLtheta}\raisebox{3.46mm}{$L_{\theta} = 4\pi aJ_1\!\left(\chi_{11}'\right)\!E_0$}\!\left[\raisebox{3.46mm}{$\cos\theta\cos\phi$}\left(\raisebox{3.46mm}{$\dfrac{{J_1}'\!\left(ka\sen\theta\right)}{1 - \left(\dfrac{ka\sen\theta}{\chi_{11}'}\right)^2}$}\right)\!\right]
\label{ec_apC:56}\\
&L_{\phi}  = -4\pi aJ_1\!\left(\chi_{11}'\right)\!E_0\!\left[\sen\phi\left(\dfrac{J_1\!\left(ka\sen\theta\right)}{ka\sen\theta}\right)\right]
\label{ec_apC:57}
\end{align}
\end{subequations}
%%%%
Introduciendo las expresiones \eqref{grup_ec_apC:14}, los campos radiados por la abertura \eqref{grup_ec_fundamentos:3} pueden expresarse como:
%%%%
\begin{subequations}
\label{grup_ec_apC:15}
\begin{align}
E_r &\simeq 0
\label{ec_apC:58}\\
E_{\theta} &\simeq j\frac{aJ_1\!\left(\chi_{11}'\right)\!kE_0e^{-jkr}}{r}\!\left[\sen\phi\left(\dfrac{J_1\!\left(ka\sen\theta\right)}{ka\sen\theta}\right)\right]
\label{ec_apC:59}\\
&\hspace{-\longitudEphi}\raisebox{3.46mm}{$E_{\phi} \simeq j\dfrac{aJ_1\!\left(\chi_{11}'\right)\!kE_0e^{-jkr}}{r}$}\!\left[\raisebox{3.46mm}{$\cos\theta\cos\phi$}\left(\raisebox{3.46mm}{$\dfrac{{J_1}'\!\left(ka\sen\theta\right)}{1 - \left(\dfrac{ka\sen\theta}{\chi_{11}'}\right)^2}$}\right)\!\right]
\label{ec_apC:60}\\
H_r &\simeq 0
\label{ec_apC:61}\\
H_\theta &\simeq -\frac{E_\phi}{\eta}
\label{ec_apC:62}\\
H_\phi &\simeq +\frac{E_\theta}{\eta}
\label{ec_apC:63}
\end{align}
\end{subequations}
%%%%
Para hallar la intensidad de radiación a partir de la expresión \eqref{ec_intro:15}, primero se determina $\lvert E_{\theta}^{\circ}\left(\theta,\phi\right)\rvert^2$ y $\lvert E_{\phi}^{\circ}\left(\theta,\phi\right)\vert^2$, cuyas expresiones son:
%%%%
\begin{subequations}
\label{grup_ec_apC:16}
\begin{align}
&\hspace{\longitudEphinor}\hspace{-\longitudEthetanor}\lvert E_{\theta}^{\circ}\left(\theta,\phi\right)\rvert^2 = \dfrac{a^2\left[J_1\!\left(\chi_{11}'\right)\right]^2k^2\left|E_0\right|^2}{r^2}\sen^2\phi\left(\dfrac{J_1\!\left(ka\sen\theta\right)}{ka\sen\theta}\right)^2
\label{ec_apC:64}\\
&\raisebox{3.46mm}{$\lvert E_{\phi}^{\circ}\left(\theta,\phi\right)\rvert^2 = \dfrac{a^2\left[J_1\!\left(\chi_{11}'\right)\right]^2k^2\left|E_0\right|^2}{r^2}$}\,\raisebox{3.46mm}{$\cos^2\theta\cos^2\phi$}\left(\raisebox{3.46mm}{$\dfrac{{J_1}'\!\left(ka\sen\theta\right)}{1 - \left(\dfrac{ka\sen\theta}{\chi_{11}'}\right)^2}$}\right)^{\!2}
\label{ec_apC:65}
\end{align}
\end{subequations}
%%%%
La expresión de la intensidad de radiación resulta:
%%%%
\begin{align}
\raisebox{3.46mm}{$U = \dfrac{a^2\left[J_1\!\left(\chi_{11}'\right)\right]^2k^2\left|E_0\right|^2}{2\eta}$}\!\left[\raisebox{3.46mm}{$\sen^2\phi\left(\dfrac{J_1\!\left(ka\sen\theta\right)}{ka\sen\theta}\right)^2 \!\!+ \cos^2\theta\cos^2\phi  $}\left(\raisebox{3.46mm}{$\dfrac{{J_1}'\!\left(ka\sen\theta\right)}{1 - \left(\dfrac{ka\sen\theta}{\chi_{11}'}\right)^2}$}\right)^{\!2}\,\right]
\label{ec_apC:66}
\end{align}
%%%%
De la expresión \eqref{ec_apC:66} se obtiene el factor de diagrama de potencia $F\left(\theta,\phi\right)$ y la constante $B_0$, por lo que:
%%%%
\begin{gather}
\raisebox{3.46mm}{$F\left(\theta,\phi\right) =\sen^2\phi\left(\dfrac{J_1\!\left(ka\sen\theta\right)}{ka\sen\theta}\right)^2 \!\!+ \cos^2\theta\cos^2\phi$}\left(\raisebox{3.46mm}{$\dfrac{{J_1}'\!\left(ka\sen\theta\right)}{1 - \left(\dfrac{ka\sen\theta}{\chi_{11}'}\right)^2}$}\right)^{\!2}
\label{ec_apC:67}\\
B_0 = \dfrac{a^2\left[J_1\!\left(\chi_{11}'\right)\right]^2k^2\left|E_0\right|^2}{2\eta}
\label{ec_apC:68}
\end{gather}
%%%%
La directividad se calcula a partir de la expresión \eqref{ec_intro:18}, empleando el factor de diagrama de potencia determinado en la expresión \eqref{ec_apC:67}.
%%%%

%%%%
\section{Guía de onda cilíndrica con extremo abierto}
\label{subsec_apendice_c_guia_cili}
%%%%

%%%%
A partir de las componentes transversales de los campos en una guía de onda cilíndrica para modo dominante, cuyas expresiones son \eqref{grup_ec_apA:29}, se deduce que los campos sobre la abertura están dados por:
%%%%
\begin{align}
\mathbf{E}_a &= 
\begin{cases} 
\makebox[0pt][l]{$E_0\,\dfrac{\sen\phi '}{\rho '}\,J_1\!\left(\dfrac{\chi '_{11}}{a}\rho '\right)\!\versor{\uprho} + E_0\cos\phi '\,{J_1}'\!\left(\dfrac{\chi '_{11}}{a}\rho '\right)\!\versor{\upphi}$}\hphantom{- \dfrac{E_0}{\eta}\cos\phi '\,{J_1}'\!\left(\dfrac{\chi '_{11}}{a}\rho '\right)\!\versor{\uprho} + \dfrac{E_0}{\eta}\dfrac{\sen\phi '}{\rho '}\,J_1\!\left(\dfrac{\chi '_{11}}{a}\rho '\right)\!\versor{\upphi}} &\text{si } \rho '\leq a\\
0 &\text{caso contrario}
\end{cases}
\label{ec_apC:69}\\
\mathbf{H}_a &= 
\begin{cases} 
- \dfrac{E_0}{\eta}\cos\phi '\,{J_1}'\!\left(\dfrac{\chi '_{11}}{a}\rho '\right)\!\versor{\uprho} + \dfrac{E_0}{\eta}\dfrac{\sen\phi '}{\rho '}\,J_1\!\left(\dfrac{\chi '_{11}}{a}\rho '\right)\!\versor{\upphi} &\text{si } \rho '\leq a\\
0  &\text{caso contrario}
\end{cases}
\label{ec_apC:70}
\end{align}
%%%%
Empleando el modelo equivalente de la Figura \ref{fig_fundamentos:10}, las densidades de corriente quedan expresadas como:
%%%%
\begin{align}
\mathbf{M}_s &=
\begin{cases}
\makebox[0pt][l]{$E_0\cos\phi '\,{J_1}'\!\left(\dfrac{\chi '_{11}}{a}\rho '\right)\!\versor{\uprho} - E_0\,\dfrac{\sen\phi '}{\rho '}\,J_1\!\left(\dfrac{\chi '_{11}}{a}\rho '\right)\!\versor{\upphi}$}\hphantom{- \dfrac{E_0}{\eta}\cos\phi '\,{J_1}'\!\left(\dfrac{\chi '_{11}}{a}\rho '\right)\!\versor{\upphi} - \dfrac{E_0}{\eta}\,\dfrac{\sen\phi '}{\rho '}\,J_1\!\left(\dfrac{\chi '_{11}}{a}\rho '\right)\!\versor{\uprho}} &\text{si } \rho '\leq a\\
0 &\text{caso contrario}
\end{cases}
\label{ec_apC:71}\\
\mathbf{J}_s &= 
\begin{cases} 
- \dfrac{E_0}{\eta}\cos\phi '\,{J_1}'\!\left(\dfrac{\chi '_{11}}{a}\rho '\right)\!\versor{\upphi} - \dfrac{E_0}{\eta}\,\dfrac{\sen\phi '}{\rho '}\,J_1\!\left(\dfrac{\chi '_{11}}{a}\rho '\right)\!\versor{\uprho} &\text{si } \rho '\leq a\\
0  &\text{caso contrario}
\end{cases}
\label{ec_apC:72}
\end{align}
%%%%
y las expresiones \eqref{grup_ec_fundamentos:6} como:
%%%%
\begin{subequations}
\label{grup_ec_apC:17}
\begin{align}
N_{\theta} &= - \dfrac{E_0}{\eta}\cos\theta\left\{\int_{0}^{\,a}\!\rho '{J_1}'\!\left(\frac{\chi '_{11}}{a}\rho '\right)\!\left[\int_{0}^{\,2\pi}\!\cos\phi '\sen\left(\phi - \phi '\right)e^{jk\rho '\sen\theta\cos\left(\phi - \phi '\right)}\,d\phi '\right]\!d\rho '\right.\notag\\
&\left.+\int_{0}^{\,a}\!J_1\!\left(\frac{\chi '_{11}}{a}\rho '\right)\!\left[\int_{0}^{\,2\pi}\!\sen\phi '\cos\left(\phi - \phi '\right)e^{jk\rho '\sen\theta\cos\left(\phi - \phi '\right)}\,d\phi '\right]\!d\rho '\right\}
\label{ec_apC:73}\\
N_{\phi} &= - \dfrac{E_0}{\eta}\left\{\int_{0}^{\,a}\!\rho '{J_1}'\!\left(\frac{\chi '_{11}}{a}\rho '\right)\!\left[\int_{0}^{\,2\pi}\!\cos\phi '\cos\left(\phi - \phi '\right)e^{jk\rho '\sen\theta\cos\left(\phi - \phi '\right)}\,d\phi '\right]\!d\rho '\right.\notag\\
&\left.-\int_{0}^{\,a}\!J_1\!\left(\frac{\chi '_{11}}{a}\rho '\right)\!\left[\int_{0}^{\,2\pi}\!\sen\phi '\sen\left(\phi - \phi '\right)e^{jk\rho '\sen\theta\cos\left(\phi - \phi '\right)}\,d\phi '\right]\!d\rho '\right\}
\label{ec_apC:74}\\
L_{\theta} &= E_0\cos\theta\left\{\int_{0}^{\,a}\!\rho '{J_1}'\!\left(\frac{\chi '_{11}}{a}\rho '\right)\!\left[\int_{0}^{\,2\pi}\!\cos\phi '\cos\left(\phi - \phi '\right)e^{jk\rho '\sen\theta\cos\left(\phi - \phi '\right)}\,d\phi '\right]\!d\rho '\right.\notag\\
&\left.-\int_{0}^{\,a}\!J_1\!\left(\frac{\chi '_{11}}{a}\rho '\right)\!\left[\int_{0}^{\,2\pi}\!\sen\phi '\sen\left(\phi - \phi '\right)e^{jk\rho '\sen\theta\cos\left(\phi - \phi '\right)}\,d\phi '\right]\!d\rho '\right\}
\label{ec_apC:75}\\
L_{\phi} &= - E_0\left\{\int_{0}^{\,a}\!\rho '{J_1}'\!\left(\frac{\chi '_{11}}{a}\rho '\right)\!\left[\int_{0}^{\,2\pi}\!\cos\phi '\sen\left(\phi - \phi '\right)e^{jk\rho '\sen\theta\cos\left(\phi - \phi '\right)}\,d\phi '\right]\!d\rho '\right.\notag\\
&\left.+\int_{0}^{\,a}\!J_1\!\left(\frac{\chi '_{11}}{a}\rho '\right)\!\left[\int_{0}^{\,2\pi}\!\sen\phi '\cos\left(\phi - \phi '\right)e^{jk\rho '\sen\theta\cos\left(\phi - \phi '\right)}\,d\phi '\right]\!d\rho '\right\}
\label{ec_apC:76}
\end{align}
\end{subequations}
%%%%
A partir de las integrales definidas:
%%%%
\begin{subequations}
\label{grup_ec_apC:18}
\begin{align}
\int_{0}^{\,2\pi}\!\cos\phi '\cos\left(\phi - \phi '\right)e^{jz\cos\left(\phi - \phi '\right)}\,d\phi ' &= \pi\cos\phi\left[J_0\!\left(z\right) - J_2\!\left(z\right)\right]
\label{ec_apC:77}\\
\int_{0}^{\,2\pi}\!\sen\phi '\sen\left(\phi - \phi '\right)e^{jz\cos\left(\phi - \phi '\right)}\,d\phi ' &= -\pi\cos\phi\left[J_0\!\left(z\right) + J_2\!\left(z\right)\right]
\label{ec_apC:78}\\
\int_{0}^{\,2\pi}\!\cos\phi '\sen\left(\phi - \phi '\right)e^{jz\cos\left(\phi - \phi '\right)}\,d\phi ' &= \pi\sen\phi\left[J_0\!\left(z\right) + J_2\!\left(z\right)\right]
\label{ec_apC:79}\\
\int_{0}^{\,2\pi}\!\sen\phi '\cos\left(\phi - \phi '\right)e^{jz\cos\left(\phi - \phi '\right)}\,d\phi ' &= \pi\sen\phi\left[J_0\!\left(z\right) - J_2\!\left(z\right)\right]
\label{ec_apC:80}
\end{align}
\end{subequations}
%%%%
las expresiones \eqref{grup_ec_apC:17} se reducen a:
%%%%
\begin{subequations}
\label{grup_ec_apC:19}
\begin{align}
\begin{split}
N_{\theta} &= - \pi\dfrac{E_0}{\eta}\cos\theta\sen\phi\left\{\int_{0}^{\,a}\!\rho '{J_1}'\!\left(\frac{\chi '_{11}}{a}\rho '\right)\!\left[J_0\!\left(k\rho '\sen\theta\right) + J_2\!\left(k\rho '\sen\theta\right)\right]d\rho '\right.\\
&\left.+\int_{0}^{\,a}\!J_1\!\left(\frac{\chi '_{11}}{a}\rho '\right)\!\left[J_0\!\left(k\rho '\sen\theta\right) - J_2\!\left(k\rho '\sen\theta\right)\right]\!d\rho '\right\}
\end{split}
\label{ec_apC:81}\\
\begin{split}
N_{\phi} &= - \pi\dfrac{E_0}{\eta}\cos\phi\left\{\int_{0}^{\,a}\!\rho '{J_1}'\!\left(\frac{\chi '_{11}}{a}\rho '\right)\!\left[J_0\!\left(k\rho '\sen\theta\right) - J_2\!\left(k\rho '\sen\theta\right)\right]d\rho '\right.\\
&\left.+\int_{0}^{\,a}\!J_1\!\left(\frac{\chi '_{11}}{a}\rho '\right)\!\left[J_0\!\left(k\rho '\sen\theta\right) + J_2\!\left(k\rho '\sen\theta\right)\right]d\rho '\right\}
\end{split}
\label{ec_apC:82}\\
\begin{split}
L_{\theta}  &= \pi E_0\cos\theta\cos\phi\left\{\int_{0}^{\,a}\!\rho '{J_1}'\!\left(\frac{\chi '_{11}}{a}\rho '\right)\!\left[J_0\!\left(k\rho '\sen\theta\right) - J_2\!\left(k\rho '\sen\theta\right)\right]d\rho '\right.\\
&\left.+\int_{0}^{\,a}\!J_1\!\left(\frac{\chi '_{11}}{a}\rho '\right)\!\left[J_0\!\left(k\rho '\sen\theta\right) + J_2\!\left(k\rho '\sen\theta\right)\right]d\rho '\right\}
\end{split}
\label{ec_apC:83}\\
\begin{split}
L_{\phi} &= - \pi E_0\sen\phi\left\{\int_{0}^{\,a}\!\rho '{J_1}'\!\left(\frac{\chi '_{11}}{a}\rho '\right)\!\left[J_0\!\left(k\rho '\sen\theta\right) + J_2\!\left(k\rho '\sen\theta\right)\right]d\rho '\right.\\
&\left.+\int_{0}^{\,a}\!J_1\!\left(\frac{\chi '_{11}}{a}\rho '\right)\!\left[J_0\!\left(k\rho '\sen\theta\right) - J_2\!\left(k\rho '\sen\theta\right)\right]d\rho '\right\}
\end{split}
\label{ec_apC:84}
\end{align}
\end{subequations}
%%%%
Empleando las relaciones de recurrencia de las funciones de Bessel:
%%%%
\begin{align}
\frac{n}{z}J_n\!\left(z\right) &= \frac{1}{2}\left[J_{n - 1}\!\left(z\right) + J_{n + 1}\!\left(z\right)\right]
\label{ec_apC:85}\\
{J_n}'\!\left(z\right) &= \frac{1}{2}\left[J_{n - 1}\!\left(z\right) - J_{n + 1}\!\left(z\right)\right]
\label{ec_apC:86}
\end{align}
%%%%
se llega a las expresiones:
%%%%
\begin{align}
{J_1}'\!\left(z\right) + \frac{J_1\!\left(z\right)}{z} &= J_0\!\left(z\right)
\label{ec_apC:87}\\
{J_1}'\!\left(z\right) - \frac{J_1\!\left(z\right)}{z} &= -J_2\!\left(z\right)
\label{ec_apC:88}
\end{align}
%%%%
por lo que $N_{\theta}$, $N_{\phi}$, $L_{\theta}$ y $L_{\phi}$ se expresan como:
%%%%
\begin{subequations}
\label{grup_ec_apC:20}
\begin{align}
\begin{split}
N_{\theta} &= - \pi\frac{\chi '_{11}}{a}\dfrac{E_0}{\eta}\cos\theta\sen\phi\left[\int_{0}^{\,a}\!\rho 'J_0\!\left(\frac{\chi '_{11}}{a}\rho '\right)\!J_0\!\left(k\rho '\sen\theta\right)d\rho '\right.\\
&-\left.\int_{0}^{\,a}\!\rho 'J_2\!\left(\frac{\chi '_{11}}{a}\rho '\right)\!J_2\!\left(k\rho '\sen\theta\right)d\rho '\right]
\end{split}
\label{ec_apC:89}\\
\begin{split}
N_{\phi} &= - \pi\frac{\chi '_{11}}{a}\dfrac{E_0}{\eta}\cos\phi\left[\int_{0}^{\,a}\!\rho 'J_0\!\left(\frac{\chi '_{11}}{a}\rho '\right)\!J_0\!\left(k\rho '\sen\theta\right)d\rho '\right.\\
&+\left.\int_{0}^{\,a}\!\rho 'J_2\!\left(\frac{\chi '_{11}}{a}\rho '\right)\!J_2\!\left(k\rho '\sen\theta\right)d\rho '\right]
\end{split}
\label{ec_apC:90}\\
\begin{split}
L_{\theta} &= \pi\frac{\chi '_{11}}{a}E_0\cos\theta\cos\phi\left[\int_{0}^{\,a}\!\rho 'J_0\!\left(\frac{\chi '_{11}}{a}\rho '\right)\!J_0\!\left(k\rho '\sen\theta\right)d\rho '\right.\\
&+\left.\int_{0}^{\,a}\!\rho 'J_2\!\left(\frac{\chi '_{11}}{a}\rho '\right)\!J_2\!\left(k\rho '\sen\theta\right)d\rho '\right]
\end{split}
\label{ec_apC:91}\\
\begin{split}
L_{\phi} &= - \pi\frac{\chi '_{11}}{a}E_0\sen\phi\left[\int_{0}^{\,a}\!\rho 'J_0\!\left(\frac{\chi '_{11}}{a}\rho '\right)\!J_0\!\left(k\rho '\sen\theta\right)d\rho '\right.\\
&-\left.\int_{0}^{\,a}\!\rho 'J_2\!\left(\frac{\chi '_{11}}{a}\rho '\right)\!J_2\!\left(k\rho '\sen\theta\right)d\rho '\right]
\end{split}
\label{ec_apC:92}
\end{align}
\end{subequations}
%%%%
Utilizando la fórmula integral de Lommel:
%%%%
\begin{align}
&\int_{0}^{\,z}\!zJ_n\!\left(\alpha z\right)\!J_n\!\left(\beta z\right)dz = \frac{z}{\alpha^2 - \beta^2}\left[J_n\!\left(\alpha z\right)\!{J_n}'\!\left(\beta z\right) - {J_n}'\!\left(\alpha z\right)\!J_n\!\left(\beta z\right)\right]
\label{ec_apC:93}
\end{align}
%%%%
es posible definir las integrales:
%%%%
\begin{subequations}
\label{grup_ec_apC:21}
\begin{align}
\int_{0}^{\,a}\!\rho 'J_0\!\left(\alpha\rho '\right)\!J_0\!\left(\beta\rho '\right)d\rho ' &= \frac{a}{\alpha^2 - \beta^2}\left[\alpha J_0\!\left(\beta a\right)\!J_1\!\left(\alpha a\right) - \beta J_0\!\left(\alpha a\right)\!J_1\!\left(\beta a\right)\right]
\label{ec_apC:94}\\
\int_{0}^{\,a}\!\rho 'J_2\!\left(\alpha\rho '\right)\!J_2\!\left(\beta\rho '\right)d\rho ' &= \frac{a}{\alpha^2 - \beta^2}\left[\beta J_1\!\left(\beta a\right)\!J_2\!\left(\alpha a\right) - \alpha J_1\!\left(\alpha a\right)\!J_2\!\left(\beta a\right)\right]
\label{ec_apC:95}
\end{align}
\end{subequations}
%%%%
Sumando y restando las integrales \eqref{grup_ec_apC:21}, se obtiene:
%%%%
\begin{subequations}
\label{grup_ec_apC:22}
\begin{align}
\int_{0}^{\,a}\!\rho '\!\left[J_0\!\left(\alpha\rho '\right)\!J_0\!\left(\beta\rho '\right) + J_2\!\left(\alpha\rho '\right)\!J_2\!\left(\beta\rho '\right)\right]d\rho ' &= \frac{2\alpha a}{\alpha^2 - \beta^2}J_1\!\left(\alpha a\right)\!{J_1}'\!\left(\beta a\right)
\label{ec_apC:96}\\
\int_{0}^{\,a}\!\rho '\!\left[J_0\!\left(\alpha\rho '\right)\!J_0\!\left(\beta\rho '\right) - J_2\!\left(\alpha\rho '\right)\!J_2\!\left(\beta\rho '\right)\right]d\rho ' &= \frac{2}{\alpha \beta}J_1\!\left(\alpha a\right)\!J_1\!\left(\beta a\right)
\label{ec_apC:97}
\end{align}
\end{subequations}
%%%%
y a partir de las expresiones \eqref{grup_ec_apC:22}, las expresiones \eqref{grup_ec_apC:20} se reducen a:
%%%%
\begin{subequations}
\label{grup_ec_apC:23}
\begin{align}
&\hspace{\longitudNphi}\hspace{-\longitudNtheta}N_{\theta}  = -2\pi aJ_1\!\left(\chi_{11}'\right)\!\dfrac{E_0}{\eta}\!\left[\cos\theta\sen\phi\left(\dfrac{J_1\!\left(ka\sen\theta\right)}{ka\sen\theta}\right)\right]
\label{ec_apC:98}\\
&\raisebox{3.46mm}{$N_{\phi} = - 2\pi aJ_1\!\left(\chi_{11}'\right)\!\dfrac{E_0}{\eta}$}\!\left[\raisebox{3.46mm}{$\cos\phi$}\left(\raisebox{3.46mm}{$\dfrac{{J_1}'\!\left(ka\sen\theta\right)}{1 - \left(\dfrac{ka\sen\theta}{\chi_{11}'}\right)^2}$}\right)\!\right]
\label{ec_apC:99}\\
&\hspace{\longitudNphi}\hspace{-\longitudLtheta}\raisebox{3.46mm}{$L_{\theta} = 2\pi aJ_1\!\left(\chi_{11}'\right)\!E_0$}\!\left[\raisebox{3.46mm}{$\cos\theta\cos\phi$}\left(\raisebox{3.46mm}{$\dfrac{{J_1}'\!\left(ka\sen\theta\right)}{1 - \left(\dfrac{ka\sen\theta}{\chi_{11}'}\right)^2}$}\right)\!\right]
\label{ec_apC:100}\\
&\hspace{\longitudNphi}\hspace{-\longitudLphi}L_{\phi}  = -2\pi aJ_1\!\left(\chi_{11}'\right)\!E_0\!\left[\sen\phi\left(\dfrac{J_1\!\left(ka\sen\theta\right)}{ka\sen\theta}\right)\right]
\label{ec_apC:101}
\end{align}
\end{subequations}
%%%%
Introduciendo las expresiones \eqref{grup_ec_apC:23}, los campos radiados por la abertura \eqref{grup_ec_fundamentos:3} pueden expresarse como:
%%%%
\begin{subequations}
\label{grup_ec_apC:24}
\begin{align}
E_r &\simeq 0
\label{ec_apC:102}\\
E_{\theta} &\simeq j\frac{aJ_1\!\left(\chi_{11}'\right)\!kE_0e^{-jkr}}{r}\!\left[\dfrac{1 + \cos\theta}{2}\sen\phi\left(\dfrac{J_1\!\left(ka\sen\theta\right)}{ka\sen\theta}\right)\right]
\label{ec_apC:103}\\
&\hspace{-\longitudEphi}\raisebox{3.46mm}{$E_{\phi} \simeq j\dfrac{aJ_1\!\left(\chi_{11}'\right)\!kE_0e^{-jkr}}{r}$}\!\left[\raisebox{3.46mm}{$\dfrac{1 + \cos\theta}{2}\cos\phi$}\left(\raisebox{3.46mm}{$\dfrac{{J_1}'\!\left(ka\sen\theta\right)}{1 - \left(\dfrac{ka\sen\theta}{\chi_{11}'}\right)^2}$}\right)\!\right]
\label{ec_apC:104}\\
H_r &\simeq 0
\label{ec_apC:105}\\
H_\theta &\simeq -\frac{E_\phi}{\eta}
\label{ec_apC:106}\\
H_\phi &\simeq +\frac{E_\theta}{\eta}
\label{ec_apC:107}
\end{align}
\end{subequations}
%%%%
Para hallar la intensidad de radiación a partir de la expresión \eqref{ec_intro:15}, primero se determina $\lvert E_{\theta}^{\circ}\left(\theta,\phi\right)\rvert^2$ y $\lvert E_{\phi}^{\circ}\left(\theta,\phi\right)\vert^2$, cuyas expresiones son:
%%%%
\begin{subequations}
\label{grup_ec_apC:25}
\begin{align}
&\hspace{\longitudEphinor}\hspace{-\longitudEthetanor}\lvert E_{\theta}^{\circ}\left(\theta,\phi\right)\rvert^2 = \dfrac{a^2\left[J_1\!\left(\chi_{11}'\right)\right]^2k^2\left|E_0\right|^2}{r^2}\left(\dfrac{1 + \cos\theta}{2}\right)^2\!\sen^2\phi\left(\dfrac{J_1\!\left(ka\sen\theta\right)}{ka\sen\theta}\right)^2
\label{ec_apC:108}\\
&\raisebox{3.46mm}{$\lvert E_{\phi}^{\circ}\left(\theta,\phi\right)\rvert^2 = \dfrac{a^2\left[J_1\!\left(\chi_{11}'\right)\right]^2k^2\left|E_0\right|^2}{r^2}$}\,\raisebox{3.46mm}{$\left(\dfrac{1 + \cos\theta}{2}\right)^2\!\cos^2\phi$}\left(\raisebox{3.46mm}{$\dfrac{{J_1}'\!\left(ka\sen\theta\right)}{1 - \left(\dfrac{ka\sen\theta}{\chi_{11}'}\right)^2}$}\right)^{\!2}
\label{ec_apC:109}
\end{align}
\end{subequations}
%%%%
La expresión de la intensidad de radiación resulta:
%%%%
\begin{align}
\begin{split}
&\raisebox{3.46mm}{$U = \dfrac{a^2\left[J_1\!\left(\chi_{11}'\right)\right]^2k^2\left|E_0\right|^2}{2\eta}\left(\dfrac{1 + \cos\theta}{2}\right)^2$}\!\left[\raisebox{3.46mm}{$\sen^2\phi\left(\dfrac{J_1\!\left(ka\sen\theta\right)}{ka\sen\theta}\right)^2$}\right.\\
&\hspace{3.06mm}\left.\raisebox{3.46mm}{$+ \cos^2\phi$}\left(\raisebox{3.46mm}{$\dfrac{{J_1}'\!\left(ka\sen\theta\right)}{1 - \left(\dfrac{ka\sen\theta}{\chi_{11}'}\right)^2}$}\right)^{\!2}\,\right]
\end{split}
\label{ec_apC:110}
\end{align}
%%%%
De la expresión \eqref{ec_apC:110} se obtiene el factor de diagrama de potencia $F\left(\theta,\phi\right)$ y la constante $B_0$, por lo que:
%%%%
\begin{gather}
\raisebox{3.46mm}{$F\left(\theta, \phi\right) = \left(\dfrac{1 + \cos\theta}{2}\right)^2$}\!\left[\raisebox{3.46mm}{$\sen^2\phi\left(\dfrac{J_1\!\left(ka\sen\theta\right)}{ka\sen\theta}\right)^2 \!\!+ \cos^2\phi$}\left(\raisebox{3.46mm}{$\dfrac{{J_1}'\!\left(ka\sen\theta\right)}{1 - \left(\dfrac{ka\sen\theta}{\chi_{11}'}\right)^2}$}\right)^{\!2}\,\right]
\label{ec_apC:111}\\
B_0 = \dfrac{a^2\left[J_1\!\left(\chi_{11}'\right)\right]^2k^2\left|E_0\right|^2}{2\eta}
\label{ec_apC:112}
\end{gather}
%%%%
La directividad se calcula a partir de la expresión \eqref{ec_intro:18}, empleando el factor de diagrama de potencia determinado en la expresión \eqref{ec_apC:111}.
%%%%

%%%%
\section{Bocina cónica}
\label{subsec_apendice_c_boci_coni}
%%%%

%%%%
A partir de las componentes transversales de los campos en una guía de onda cilíndrica para modo dominante, cuyas expresiones son \eqref{grup_ec_apA:29}, y considerando que la fase no es uniforme sobre la abertura, se deduce que los campos sobre la abertura están dados por:
%%%%
\begin{align}
\mathbf{E}_a &= 
\begin{cases} 
\makebox[0pt][l]{$\left[E_0\,\dfrac{\sen\phi '}{\rho '}\,J_1\!\left(\dfrac{\chi '_{11}}{A}\rho '\right)\!\versor{\uprho} + E_0\cos\phi '\,{J_1}'\!\left(\dfrac{\chi '_{11}}{A}\rho '\right)\!\versor{\upphi}\right]\!e^{-jk\delta\left(\rho '\right)}$}\hphantom{\left[- \dfrac{E_0}{\eta}\cos\phi '\,{J_1}'\!\left(\dfrac{\chi '_{11}}{A}\rho '\right)\!\versor{\uprho} + \dfrac{E_0}{\eta}\dfrac{\sen\phi '}{\rho '}\,J_1\!\left(\dfrac{\chi '_{11}}{A}\rho '\right)\!\versor{\upphi}\right]\!e^{-jk\delta\left(\rho '\right)}} &\text{si } \rho '\leq A\\
0 &\text{caso contrario}
\end{cases}
\label{ec_apC:113}\\
\mathbf{H}_a &= 
\begin{cases} 
\left[- \dfrac{E_0}{\eta}\cos\phi '\,{J_1}'\!\left(\dfrac{\chi '_{11}}{A}\rho '\right)\!\versor{\uprho} + \dfrac{E_0}{\eta}\dfrac{\sen\phi '}{\rho '}\,J_1\!\left(\dfrac{\chi '_{11}}{A}\rho '\right)\!\versor{\upphi}\right]\!e^{-jk\delta\left(\rho '\right)} &\text{si } \rho '\leq A\\
0  &\text{caso contrario}
\end{cases}
\label{ec_apC:114}
\end{align}
%%%%
donde el término $e^{-jk\delta\left(\rho '\right)}$ representa la variación de fase de los campos sobre la abertura en la dirección $\rho$.

Empleando el modelo equivalente de la Figura \ref{fig_fundamentos:10}, las densidades de corriente quedan expresadas como:
%%%%
\begin{align}
\mathbf{M}_s &=
\begin{cases}
\makebox[0pt][l]{$\left[E_0\cos\phi '\,{J_1}'\!\left(\dfrac{\chi '_{11}}{A}\rho '\right)\!\versor{\uprho} - E_0\,\dfrac{\sen\phi '}{\rho '}\,J_1\!\left(\dfrac{\chi '_{11}}{A}\rho '\right)\!\versor{\upphi}\right]\!e^{-jk\delta\left(\rho '\right)}$}\hphantom{\left[- \dfrac{E_0}{\eta}\cos\phi '\,{J_1}'\!\left(\dfrac{\chi '_{11}}{A}\rho '\right)\!\versor{\upphi} - \dfrac{E_0}{\eta}\,\dfrac{\sen\phi '}{\rho '}\,J_1\!\left(\dfrac{\chi '_{11}}{A}\rho '\right)\!\versor{\uprho}\right]\!e^{-jk\delta\left(\rho '\right)}} &\text{si } \rho '\leq A\\
0 &\text{caso contrario}
\end{cases}
\label{ec_apC:115}\\
\mathbf{J}_s &= 
\begin{cases} 
\left[- \dfrac{E_0}{\eta}\cos\phi '\,{J_1}'\!\left(\dfrac{\chi '_{11}}{A}\rho '\right)\!\versor{\upphi} - \dfrac{E_0}{\eta}\,\dfrac{\sen\phi '}{\rho '}\,J_1\!\left(\dfrac{\chi '_{11}}{A}\rho '\right)\!\versor{\uprho}\right]\!e^{-jk\delta\left(\rho '\right)} &\text{si } \rho '\leq A\\
0  &\text{caso contrario}
\end{cases}
\label{ec_apC:116}
\end{align}
%%%%
y las expresiones \eqref{grup_ec_fundamentos:6} como:
%%%%
\begin{subequations}
\label{grup_ec_apC:26}
\begin{flalign}
N_{\theta} &= - \dfrac{E_0}{\eta}\left\{\int_{0}^{\,A}\!\rho '{J_1}'\!\left(\frac{\chi '_{11}}{a}\rho '\right)\!e^{-jk\delta\left(\rho '\right)}\!\left[\int_{0}^{\,2\pi}\!\cos\phi '\sen\left(\phi - \phi '\right)e^{jk\rho '\sen\theta\cos\left(\phi - \phi '\right)}\,d\phi '\right]\!d\rho '\right.\notag\\
&\left.+\int_{0}^{\,A}\!J_1\!\left(\frac{\chi '_{11}}{a}\rho '\right)\!e^{-jk\delta\left(\rho '\right)}\!\left[\int_{0}^{\,2\pi}\!\sen\phi '\cos\left(\phi - \phi '\right)e^{jk\rho '\sen\theta\cos\left(\phi - \phi '\right)}\,d\phi '\right]\!d\rho '\right\}\cos\theta
\label{ec_apC:117}\\
N_{\phi} &= - \dfrac{E_0}{\eta}\left\{\int_{0}^{\,A}\!\rho '{J_1}'\!\left(\frac{\chi '_{11}}{a}\rho '\right)\!e^{-jk\delta\left(\rho '\right)}\!\left[\int_{0}^{\,2\pi}\!\cos\phi '\cos\left(\phi - \phi '\right)e^{jk\rho '\sen\theta\cos\left(\phi - \phi '\right)}\,d\phi '\right]\!d\rho '\right.\notag\\
&\left.-\int_{0}^{\,A}\!J_1\!\left(\frac{\chi '_{11}}{a}\rho '\right)\!e^{-jk\delta\left(\rho '\right)}\!\left[\int_{0}^{\,2\pi}\!\sen\phi '\sen\left(\phi - \phi '\right)e^{jk\rho '\sen\theta\cos\left(\phi - \phi '\right)}\,d\phi '\right]\!d\rho '\right\}
\label{ec_apC:118}\\
L_{\theta} &= E_0\left\{\int_{0}^{\,A}\!\rho '{J_1}'\!\left(\frac{\chi '_{11}}{a}\rho '\right)\!e^{-jk\delta\left(\rho '\right)}\!\left[\int_{0}^{\,2\pi}\!\cos\phi '\cos\left(\phi - \phi '\right)e^{jk\rho '\sen\theta\cos\left(\phi - \phi '\right)}\,d\phi '\right]\!d\rho '\right.\notag\\
&\left.-\int_{0}^{\,A}\!J_1\!\left(\frac{\chi '_{11}}{a}\rho '\right)\!e^{-jk\delta\left(\rho '\right)}\!\left[\int_{0}^{\,2\pi}\!\sen\phi '\sen\left(\phi - \phi '\right)e^{jk\rho '\sen\theta\cos\left(\phi - \phi '\right)}\,d\phi '\right]\!d\rho '\right\}\cos\theta
\label{ec_apC:119}\\
L_{\phi} &= - E_0\left\{\int_{0}^{\,A}\!\rho '{J_1}'\!\left(\frac{\chi '_{11}}{a}\rho '\right)\!e^{-jk\delta\left(\rho '\right)}\!\left[\int_{0}^{\,2\pi}\!\cos\phi '\sen\left(\phi - \phi '\right)e^{jk\rho '\sen\theta\cos\left(\phi - \phi '\right)}\,d\phi '\right]\!d\rho '\right.\notag\\
&\left.+\int_{0}^{\,A}\!J_1\!\left(\frac{\chi '_{11}}{a}\rho '\right)\!e^{-jk\delta\left(\rho '\right)}\!\left[\int_{0}^{\,2\pi}\!\sen\phi '\cos\left(\phi - \phi '\right)e^{jk\rho '\sen\theta\cos\left(\phi - \phi '\right)}\,d\phi '\right]\!d\rho '\right\}
\label{ec_apC:120}
\end{flalign}
\end{subequations}
%%%%
A partir de las integrales definidas:
%%%%
\begin{subequations}
\label{grup_ec_apC:27}
\begin{align}
\int_{0}^{\,2\pi}\!\cos\phi '\cos\left(\phi - \phi '\right)e^{jz\cos\left(\phi - \phi '\right)}\,d\phi ' &= \pi\cos\phi\left[J_0\!\left(z\right) - J_2\!\left(z\right)\right]
\label{ec_apC:121}\\
\int_{0}^{\,2\pi}\!\sen\phi '\sen\left(\phi - \phi '\right)e^{jz\cos\left(\phi - \phi '\right)}\,d\phi ' &= -\pi\cos\phi\left[J_0\!\left(z\right) + J_2\!\left(z\right)\right]
\label{ec_apC:122}\\
\int_{0}^{\,2\pi}\!\cos\phi '\sen\left(\phi - \phi '\right)e^{jz\cos\left(\phi - \phi '\right)}\,d\phi ' &= \pi\sen\phi\left[J_0\!\left(z\right) + J_2\!\left(z\right)\right]
\label{ec_apC:123}\\
\int_{0}^{\,2\pi}\!\sen\phi '\cos\left(\phi - \phi '\right)e^{jz\cos\left(\phi - \phi '\right)}\,d\phi ' &= \pi\sen\phi\left[J_0\!\left(z\right) - J_2\!\left(z\right)\right]
\label{ec_apC:124}
\end{align}
\end{subequations}
%%%%
las expresiones \eqref{grup_ec_apC:26} se reducen a:
%%%%
\begin{subequations}
\label{grup_ec_apC:28}
\begin{align}
N_{\theta} &= - \pi\dfrac{E_0}{\eta}\cos\theta\sen\phi\left\{\int_{0}^{\,A}\!\rho '{J_1}'\!\left(\frac{\chi '_{11}}{a}\rho '\right)\!e^{-jk\delta\left(\rho '\right)}\!\left[J_0\!\left(k\rho '\sen\theta\right) + J_2\!\left(k\rho '\sen\theta\right)\right]d\rho '\right.\notag\\
&\left.+\int_{0}^{\,A}\!J_1\!\left(\frac{\chi '_{11}}{a}\rho '\right)\!e^{-jk\delta\left(\rho '\right)}\!\left[J_0\!\left(k\rho '\sen\theta\right) - J_2\!\left(k\rho '\sen\theta\right)\right]\!d\rho '\right\}
\label{ec_apC:125}\\
N_{\phi} &= - \pi\dfrac{E_0}{\eta}\cos\phi\left\{\int_{0}^{\,A}\!\rho '{J_1}'\!\left(\frac{\chi '_{11}}{a}\rho '\right)\!e^{-jk\delta\left(\rho '\right)}\!\left[J_0\!\left(k\rho '\sen\theta\right) - J_2\!\left(k\rho '\sen\theta\right)\right]d\rho '\right.\notag\\
&\left.+\int_{0}^{\,A}\!J_1\!\left(\frac{\chi '_{11}}{a}\rho '\right)\!e^{-jk\delta\left(\rho '\right)}\!\left[J_0\!\left(k\rho '\sen\theta\right) + J_2\!\left(k\rho '\sen\theta\right)\right]d\rho '\right\}
\label{ec_apC:126}\\
L_{\theta}  &= \pi E_0\cos\theta\cos\phi\left\{\int_{0}^{\,A}\!\rho '{J_1}'\!\left(\frac{\chi '_{11}}{a}\rho '\right)\!e^{-jk\delta\left(\rho '\right)}\!\left[J_0\!\left(k\rho '\sen\theta\right) - J_2\!\left(k\rho '\sen\theta\right)\right]d\rho '\right.\notag\\
&\left.+\int_{0}^{\,A}\!J_1\!\left(\frac{\chi '_{11}}{a}\rho '\right)\!e^{-jk\delta\left(\rho '\right)}\!\left[J_0\!\left(k\rho '\sen\theta\right) + J_2\!\left(k\rho '\sen\theta\right)\right]d\rho '\right\}
\label{ec_apC:127}\\
L_{\phi} &= - \pi E_0\sen\phi\left\{\int_{0}^{\,A}\!\rho '{J_1}'\!\left(\frac{\chi '_{11}}{a}\rho '\right)\!e^{-jk\delta\left(\rho '\right)}\!\left[J_0\!\left(k\rho '\sen\theta\right) + J_2\!\left(k\rho '\sen\theta\right)\right]d\rho '\right.\notag\\
&\left.+\int_{0}^{\,A}\!J_1\!\left(\frac{\chi '_{11}}{a}\rho '\right)\!e^{-jk\delta\left(\rho '\right)}\!\left[J_0\!\left(k\rho '\sen\theta\right) - J_2\!\left(k\rho '\sen\theta\right)\right]d\rho '\right\}
\label{ec_apC:128}
\end{align}
\end{subequations}
%%%%
Empleando las relaciones de recurrencia de las funciones de Bessel:
%%%%
\begin{align}
\frac{n}{z}J_n\!\left(z\right) &= \frac{1}{2}\left[J_{n - 1}\!\left(z\right) + J_{n + 1}\!\left(z\right)\right]
\label{ec_apC:129}\\
{J_n}'\!\left(z\right) &= \frac{1}{2}\left[J_{n - 1}\!\left(z\right) - J_{n + 1}\!\left(z\right)\right]
\label{ec_apC:130}
\end{align}
%%%%
se llega a las expresiones:
%%%%
\begin{align}
{J_1}'\!\left(z\right) + \frac{J_1\!\left(z\right)}{z} &= J_0\!\left(z\right)
\label{ec_apC:131}\\
{J_1}'\!\left(z\right) - \frac{J_1\!\left(z\right)}{z} &= -J_2\!\left(z\right)
\label{ec_apC:132}
\end{align}
%%%%
por lo que $N_{\theta}$, $N_{\phi}$, $L_{\theta}$ y $L_{\phi}$ se expresan como:
%%%%
\begin{subequations}
\label{grup_ec_apC:29}
\begin{align}
N_{\theta} &= - \pi\frac{\chi '_{11}}{a}\dfrac{E_0}{\eta}\left[\,\cos\theta\sen\phi\left(\,\mathbf{I}_1 - \mathbf{I}_2\right)\right]
\label{ec_apC:133}\\
N_{\phi} &= - \pi\frac{\chi '_{11}}{a}\dfrac{E_0}{\eta}\left[\,\cos\phi\left(\,\mathbf{I}_1 + \mathbf{I}_2\right)\right]
\label{ec_apC:134}\\
L_{\theta} &= \pi\frac{\chi '_{11}}{a}E_0\left[\,\cos\theta\cos\phi\left(\,\mathbf{I}_1 +\mathbf{I}_2\right)\right]
\label{ec_apC:135}\\
L_{\phi} &= - \pi\frac{\chi '_{11}}{a}E_0\left[\,\sen\phi\left(\,\mathbf{I}_1 - \mathbf{I}_2\right)\right]
\label{ec_apC:136}
\end{align}
\end{subequations}
%%%%
donde:
%%%%
\begin{align}
\mathbf{I}_1 &= \int_{0}^{\,A}\!\rho 'J_0\!\left(\frac{\chi '_{11}}{a}\rho '\right)\!J_0\!\left(k\rho '\sen\theta\right)\!e^{-jk\delta\left(\rho '\right)}d\rho '
\label{ec_apC:137}\\
\mathbf{I}_2 &= \int_{0}^{\,A}\!\rho 'J_2\!\left(\frac{\chi '_{11}}{a}\rho '\right)\!J_2\!\left(k\rho '\sen\theta\right)\!e^{-jk\delta\left(\rho '\right)}d\rho '
\label{ec_apC:138}
\end{align}
%%%%
Las integrales $\mathbf{I}_1$ y $\mathbf{I}_2$  se resuelve numéricamente, expresando las diferencias de caminos en la dirección $\rho$ $\delta\left(\rho '\right)$ como:
%%%%
\begin{align}
\delta\left(\rho '\right) = R - R_0 = \sqrt{{R_0}^2 + {\rho '}^2} - R_0
\label{ec_apC:139}
\end{align}
%%%%
Introduciendo las expresiones \eqref{grup_ec_apC:29}, los campos radiados por la abertura \eqref{grup_ec_fundamentos:3} pueden expresarse como:
%%%%
\begin{subequations}
\label{grup_ec_apC:30}
\begin{align}
E_r &\simeq 0
\label{ec_apC:140}\\
E_{\theta} &\simeq j\dfrac{k\chi '_{11}E_0e^{-jkr}}{2ar}\left[\dfrac{1 + \cos\theta}{2}\sen\phi\left(\,\mathbf{I}_1 - \mathbf{I}_2\right)\right]
\label{ec_apC:141}\\
E_{\phi} &\simeq j\dfrac{k\chi '_{11}E_0e^{-jkr}}{2ar}\left[\dfrac{1 + \cos\theta}{2}\cos\phi\left(\,\mathbf{I}_1 + \mathbf{I}_2\right)\right]
\label{ec_apC:142}\\
H_r &\simeq 0
\label{ec_apC:143}\\
H_\theta &\simeq -\frac{E_\phi}{\eta}
\label{ec_apC:144}\\
H_\phi &\simeq +\frac{E_\theta}{\eta}
\label{ec_apC:145}
\end{align}
\end{subequations}
%%%%
Para hallar la intensidad de radiación a partir de la expresión \eqref{ec_intro:15}, primero se determina $\lvert E_{\theta}^{\circ}\left(\theta,\phi\right)\rvert^2$ y $\lvert E_{\phi}^{\circ}\left(\theta,\phi\right)\vert^2$, cuyas expresiones son:
%%%%
\begin{subequations}
\label{grup_ec_apC:31}
\begin{align}
\lvert E_{\theta}^{\circ}\left(\theta,\phi\right)\rvert^2 &= \frac{k^2{\chi '_{11}}^2\left|E_0\right|^2}{4a^2r^2}\left(\dfrac{1 + \cos\theta}{2}\right)^2\!\sen^2\phi\left|\,\mathbf{I}_1 - \mathbf{I}_2\right|^2
\label{ec_apC:146}\\
\lvert E_{\phi}^{\circ}\left(\theta,\phi\right)\rvert^2 &= \frac{k^2{\chi '_{11}}^2\left|E_0\right|^2}{4a^2r^2}\left(\dfrac{1 + \cos\theta}{2}\right)^2\!\cos^2\phi\left|\,\mathbf{I}_1 + \mathbf{I}_2\right|^2
\label{ec_apC:147}
\end{align}
\end{subequations}
%%%%
La expresión de la intensidad de radiación resulta:
%%%%
\begin{align}
U = \frac{k^2{\chi '_{11}}^2\left|E_0\right|^2}{8a^2\eta}\left(\dfrac{1 + \cos\theta}{2}\right)^2\!\left(\sen^2\phi\left|\,\mathbf{I}_1 - \mathbf{I}_2\right|^2 + \cos^2\phi\left|\,\mathbf{I}_1 + \mathbf{I}_2\right|^2\right)
\label{ec_apC:148}
\end{align}
%%%%
De la expresión \eqref{ec_apC:148} se obtiene el factor de diagrama de potencia $F\left(\theta,\phi\right)$ y la constante $B_0$, por lo que:
%%%%
\begin{gather}
F\left(\theta,\phi\right) = \left(\dfrac{1 + \cos\theta}{2}\right)^2\!\left(\sen^2\phi\left|\,\mathbf{I}_1 - \mathbf{I}_2\right|^2 + \cos^2\phi\left|\,\mathbf{I}_1 + \mathbf{I}_2\right|^2\right)
\label{ec_apC:149}\\
B_0 = \frac{k^2{\chi '_{11}}^2\left|E_0\right|^2}{8a^2\eta}
\label{ec_apC:150}
\end{gather}
%%%%
La directividad se calcula a partir de la expresión \eqref{ec_intro:18}, empleando el factor de diagrama de potencia determinado en la expresión \eqref{ec_apC:149}.
%%%%