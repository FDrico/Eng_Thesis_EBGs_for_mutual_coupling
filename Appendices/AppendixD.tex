%%%%
\section{Expresión del versor \texorpdfstring{$\versor{e}_i$}{text}}
%%%%

Los versores $\versor{e}_i$ y $\versor{r}$ se expresan como:
%%%%
\begin{align}
\versor{e}_i &= {\text{e}_i}_x\,\versor{x} + {\text{e}_i}_y\,\versor{y} + {\text{e}_i}_z\,\versor{z}
\label{ec_apD:1}\\
\versor{r}' &= {\text{r}'}_{\!x}\,\versor{x} + {\text{r}'}_{\!y}\,\versor{y} + {\text{r}'}_{\!z}\,\versor{z}
\label{ec_apD:2}
\end{align}
%%%%
donde:
%%%%
\begin{subequations}
\label{grup_ec_apD:1}
\begin{align}
{\text{r}'}_{\!x} &= \sen\theta '\cos\phi '
\label{ec_apD:3}\\
{\text{r}'}_{\!y} &= \sen\theta '\sen\phi '
\label{ec_apD:4}\\
{\text{r}'}_{\!z} &= \cos\theta '
\label{ec_apD:5}
\end{align}
\end{subequations}
%%%%
Como $\versor{e}_i$ es perpendicular a $\versor{r}$, se obtiene la relación:
%%%%
\begin{align}
\versor{r}'\prodesc\versor{e}_i = 0 \Longrightarrow {\text{r}'}_{\!x}\,{\text{e}_i}_x + {\text{r}'}_{\!y}\,{\text{e}_i}_y + {\text{r}'}_{\!z}\,{\text{e}_i}_z = 0
\label{ec_apD:6}
\end{align}
%%%%
Como $\versor{e}_i$ está contenido en el plano formado por $\versor{r}'$ y $\versor{y}$, es necesario hallar el vector normal a dicho plano, cuya expresión es:
%%%%
\begin{align}
\mathbf{N} = \versor{r}'\prodvec\versor{y} =
\begin{vmatrix}
\versor{x} & \versor{y} & \versor{z}\\
{\text{r}'}_{\!x} & {\text{r}'}_{\!y} & {\text{r}'}_{\!z}\\
0 & 1 & 0\\
\end{vmatrix}
= - {\text{r}'}_{\!z}\,\versor{x} + {\text{r}'}_{\!x}\,\versor{z} \Longrightarrow \mathbf{N} = \left(-{\text{r}'}_{\!z}\,,0\,,{\text{r}'}_{\!x}\right)
\label{ec_apD:7}
\end{align}
%%%%
Como el punto $\left(0\,,0\,,0\right)$ pertenece al plano formado por $\versor{r}'$ y $\versor{y}$, su ecuación resultante es:
%%%%
\begin{align}
\left(-{\text{r}'}_{\!z}\,,0\,,{\text{r}'}_{\!x}\right)\left(x\,,y\,,z\right) = 0 \Longrightarrow -{\text{r}'}_{\!z}\,x + {\text{r}'}_{\!x}\,z = 0
\label{ec_apD:8}
\end{align}
%%%%
Dado que $\versor{e}_i$ tiene norma unitaria y está contenido en el plano, se obtienen las relaciones:
%%%%
\begin{gather}
{{\text{e}_i}_x}^2 + {{\text{e}_i}_y}^2 + {{\text{e}_i}_z}^2 = 1
\label{ec_apD:9}\\
-{\text{r}'}_{\!z}\,{\text{e}_i}_x + {\text{r}'}_{\!x}\,{\text{e}_i}_z = 0
\label{ec_apD:10}
\end{gather}
%%%%
Empleando la expresión \eqref{ec_apD:10}, se obtiene la relación:
%%%%
\begin{align}
{\text{e}_i}_z = \dfrac{{\text{r}'}_{\!z}}{{\text{r}'}_{\!x}}\,{\text{e}_i}_x
\label{ec_apD:11}
\end{align}
%%%%
y utilizando la relación \eqref{ec_apD:11}, la expresión \eqref{ec_apD:6} se reduce a:
%%%%
\begin{align}
{\text{r}'}_{\!x}\,{\text{e}_i}_x + {\text{r}'}_{\!y}\,{\text{e}_i}_y + \dfrac{{{\text{r}'}_{\!z}}^2}{{\text{r}'}_{\!x}}\,{\text{e}_i}_x = 0
\label{ec_apD:12}
\end{align}
%%%%
Trabajando la expresión \eqref{ec_apD:12}, se obtiene:
%%%%
\begin{align}
{\text{r}'}_{\!x}\,{\text{e}_i}_x + {\text{r}'}_{\!y}\,{\text{e}_i}_y + \dfrac{{{\text{r}'}_{\!z}}^2}{{\text{r}'}_{\!x}}\,{\text{e}_i}_x = 0
\label{ec_apD:13}
\end{align}
%%%%
Desarrollando la expresión \eqref{ec_apD:13}, se llega a:
%%%%
\begin{align}
{\text{e}_i}_y  = -\left(\dfrac{{\text{r}'}_{\!x}}{{\text{r}'}_{\!y}} + \dfrac{{{\text{r}'}_{\!z}}^2}{{\text{r}'}_{\!x}\,{\text{r}'}_{\!y}}\right){\text{e}_i}_x
\label{ec_apD:14}
\end{align}
%%%%
La expresión \eqref{ec_apD:9} se reduce, empleando las expresiones \eqref{ec_apD:11} y \eqref{ec_apD:14}, a:
%%%%
\begin{align}
{{\text{e}_i}_x}^2 + \left(\dfrac{{\text{r}'}_{\!x}}{{\text{r}'}_{\!y}} + \dfrac{{{\text{r}'}_{\!z}}^2}{{\text{r}'}_{\!x}\,{\text{r}'}_{\!y}}\right)^2\!{{\text{e}_i}_x}^2 + \left(\dfrac{{\text{r}'}_{\!z}}{{\text{r}'}_{\!x}}\right)^2\!{{\text{e}_i}_x}^2 = 1
\label{ec_apD:15}
\end{align}
%%%%
Trabajando la expresión \eqref{ec_apD:15}, se llega a:
%%%%
\begin{align}
{{\text{e}_i}_x}^2 = \dfrac{{{\text{r}'}_{\!x}}^2\,{{\text{r}'}_{\!y}}^2}{\left({{\text{r}'}_{\!x}}^2 + {{\text{r}'}_{\!z}}^2\right)\left(  {{\text{r}'}_{\!x}}^2 + {{\text{r}'}_{\!y}}^2 + {{\text{r}'}_{\!z}}^2\right)}
\label{ec_apD:16}
\end{align}
%%%%
Como $\versor{r}'$ tiene norma unitaria, se cumple que:
%%%%
\begin{align}
{{\text{r}'}_x}^2 + {{\text{r}'}_y}^2 + {{\text{r}'}_z}^2 = 1
\label{ec_apD:17}
\end{align}
%%%%
y la expresión \eqref{ec_apD:16} se reduce a:
%%%%
\begin{align}
{{\text{e}_i}_x}^2 = \dfrac{{{\text{r}'}_{\!x}}^2\,{{\text{r}'}_{\!y}}^2}{\big(\underbrace{{{\text{r}'}_{\!x}}^2 + {{\text{r}'}_{\!y}}^2 + {{\text{r}'}_{\!z}}^2}_{1} - {{\text{r}'}_{\!y}}^2\big)\big(  \underbrace{{{\text{r}'}_{\!x}}^2 + {{\text{r}'}_{\!y}}^2 + {{\text{r}'}_{\!z}}^2}_{1}\big)} = \dfrac{{{\text{r}'}_{\!x}}^2\,{{\text{r}'}_{\!y}}^2}{1 - {{\text{r}'}_{\!y}}^2}
\label{ec_apD:18}
\end{align}
%%%%
Empleando las expresiones \eqref{grup_ec_apD:1}, la expresión resultante de ${\text{e}_i}_x$ es:
%%%%
\begin{align}
{\text{e}_i}_x = \pm\dfrac{\sen^2\theta '\sen\phi '\cos\phi '}{\sqrt{1 - \sen^2\theta '\sen^2\phi '}}
\label{ec_apD:19}
\end{align}
%%%%
Para determinar el signo de ${\text{e}_i}_x$, se tiene como condición que:
%%%%
\begin{align}
\versor{e}_i = \versor{y}\text{ , si }\theta ' = 0\text{ y }\phi ' = 0
\label{ec_apD:20}
\end{align}
%%%%
Tomando ${\text{e}_i}_x$ con signo negativo, las expresiones resultantes de las componentes del versor $\versor{e}_i$ son:
%%%%
\begin{subequations}
\label{grup_ec_apD:2}
\begin{align}
{\text{e}_i}_x &= -\dfrac{\sen^2\theta '\sen\phi '\cos\phi '}{\sqrt{1 - \sen^2\theta '\sen^2\phi '}}
\label{ec_apD:21}\\
{\text{e}_i}_y &= \dfrac{\sen^2\theta '\cos^2\phi ' + \cos^2\theta '}{\sqrt{1 - \sen^2\theta '\sen^2\phi '}}
\label{ec_apD:22}\\
{\text{e}_i}_z &= -\dfrac{\sen\theta '\cos\theta '\sen\phi '}{\sqrt{1 - \sen^2\theta '\sen^2\phi '}}
\label{ec_apD:23}
\end{align}
\end{subequations}
%%%%
y para $\theta ' = 0$ y $\phi ' = 0$, las componentes de $\versor{e}_i$ se reducen a:
%%%%
\begin{align*}
{\text{e}_i}_x = 0\hspace{10mm}{\text{e}_i}_y = 1\hspace{10mm}{\text{e}_i}_z = 0
\end{align*}
%%%%
cumpliéndose entonces la condición \eqref{ec_apD:20} y quedando $\versor{e}_i$ expresado como:
%%%%
\begin{align}
\versor{e}_i= -\dfrac{\sen^2\theta '\sen\phi '\cos\phi '}{\sqrt{1 - \sen^2\theta '\sen^2\phi '}}\versor{x} + \dfrac{\sen^2\theta '\cos^2\phi ' + \cos^2\theta'}{\sqrt{1 - \sen^2\theta '\sen^2\phi '}}\versor{y} - \dfrac{\sen\theta '\cos\theta '\sen\phi '}{\sqrt{1 - \sen^2\theta '\sen^2\phi '}}\versor{z}
\label{ec_apD:24}
\end{align}
%%%%

%%%%
\section{Expresión del vector \texorpdfstring{$\mathbf{u}$}{text}}
%%%%

El vector $\mathbf{u}$, determinado en la Sección \ref{sec_principios_cam_rad}, se expresa como:
%%%%
\begin{align}
\mathbf{u} = \left(\versor{n}\prodesc\versor{e}_i\right)\versor{r}' - \left(\versor{n}\prodesc\versor{r}'\right)\versor{e}_i
\label{ec_apD:25}
\end{align}
%%%%
mientras que la expresión de $\versor{n}$ es:
%%%%
\begin{align}
\versor{n} = -\cos\left(\frac{\theta '}{2}\right)\!\versor{r}' + \sen\left(\frac{\theta '}{2}\right)\!\versor{\uptheta}'
\label{ec_apD:26}
\end{align}
%%%%
Para determinar $\versor{n}\prodesc\versor{e}_i$, primero se expresa $\versor{n}$ en coordenadas cartesianas. Los versores $\versor{r}'$ y $\versor{\uptheta}'$ se expresan como:
%%%%
\begin{subequations}
\label{grup_ec_apD:3}
\begin{align}
\versor{r}' &= \sen\theta '\cos\phi '\,\versor{x} + \sen\theta '\sen\phi '\,\versor{y} + \cos\theta '\,\versor{z}
\label{ec_apD:27}\\
\versor{\uptheta}' &= \cos\theta '\cos\phi '\,\versor{x} + \cos\theta '\sen\phi '\,\versor{y} - \sen\theta '\,\versor{z}
\label{ec_apD:28}
\end{align}
\end{subequations}
%%%%
y, empleando las expresiones \eqref{grup_ec_apD:3}, $\versor{n}$ resulta:
%%%%
\begin{align}
\versor{n} = - \sen\left(\dfrac{\theta '}{2}\right)\cos\phi '\,\versor{x} - \sen\left(\dfrac{\theta '}{2}\right)\sen\phi '\,\versor{y} - \cos\left(\dfrac{\theta '}{2}\right)\versor{z}
\label{ec_apD:29}
\end{align}
%%%%
$\versor{n}\prodesc\versor{e}_i$ queda expresado como:
%%%%
\begin{align}
\begin{split}
\versor{n}\prodesc\versor{e}_i &= \dfrac{\sen^2\theta '\sen\left(\dfrac{\theta '}{2}\right)\sen\phi '\cos^2\phi '}{\sqrt{1 - \sen^2\theta '\sen^2\phi '}} + \dfrac{\sen\theta '\cos\theta '\cos\left(\dfrac{\theta '}{2}\right)\sen\phi '}{\sqrt{1 - \sen^2\theta '\sen^2\phi '}}\\
&- \dfrac{\sen\left(\dfrac{\theta '}{2}\right)\sen\phi '\left[\sen^2\theta '\cos^2\phi ' + \cos^2\theta '\right]}{\sqrt{1 - \sen^2\theta '\sen^2\phi '}}
\end{split}
\label{ec_apD:30}
\end{align}
%%%%
Desarrollando la expresión \eqref{ec_apD:30}, se llega a:
%%%%
\begin{align}
\versor{n}\prodesc\versor{e}_i &= \dfrac{\cos\theta '\sen\left(\dfrac{\theta '}{2}\right)\sen\phi '}{\sqrt{1 - \sen^2\theta '\sen^2\phi '}}
\label{ec_apD:31}
\end{align}
%%%%
y $\left(\versor{n}\prodesc\versor{e}_i\right)\versor{r}'$ resulta:
%%%%
\begin{align}
\begin{split}
\left(\versor{n}\prodesc\versor{e}_i\right)\versor{r}' &= \dfrac{\sen\theta '\cos\theta '\sen\left(\dfrac{\theta '}{2}\right)\sen\phi '\cos\phi '}{\sqrt{1 - \sen^2\theta '\sen^2\phi '}}\versor{x} + \dfrac{\sen\theta '\cos\theta '\sen\left(\dfrac{\theta '}{2}\right)\sen^2\phi '}{\sqrt{1 - \sen^2\theta '\sen^2\phi '}}\versor{y}\\
&+ \dfrac{\cos^2\theta '\sen\left(\dfrac{\theta '}{2}\right)\sen\phi '}{\sqrt{1 - \sen^2\theta '\sen^2\phi '}}\versor{z}
\end{split}
\label{ec_apD:32}
\end{align}
%%%%
$\versor{n}\prodesc\versor{r}'$ queda expresado como:
%%%%
\begin{align}
\versor{n}\prodesc\versor{r}' &= \left[-\cos\left(\frac{\theta '}{2}\right)\!\versor{r}' + \sen\left(\frac{\theta '}{2}\right)\!\versor{\uptheta}'\right]\!\prodesc\versor{r}' = -\cos\left(\frac{\theta '}{2}\right)
\label{ec_apD:33}
\end{align}
%%%%
y $\left(\versor{n}\prodesc\versor{r}'\right)\versor{e}_i$ resulta:
%%%%
\begin{align}
\left(\versor{n}\prodesc\versor{r}'\right)\versor{e}_i' &= \dfrac{\sen^2\theta '\cos\left(\dfrac{\theta '}{2}\right)\sen\phi '\cos\phi '}{\sqrt{1 - \sen^2\theta '\sen^2\phi '}}\versor{x} - \dfrac{\cos\left(\dfrac{\theta '}{2}\right)\!\left[\sen^2\theta '\cos^2\phi ' + \cos^2\theta '\right]}{\sqrt{1 - \sen^2\theta '\sen^2\phi '}}\versor{y}\notag\\
&+ \dfrac{\sen\theta '\cos\theta '\cos\left(\dfrac{\theta '}{2}\right)\sen\phi '}{\sqrt{1 - \sen^2\theta '\sen^2\phi '}}\versor{z}
\label{ec_apD:34}
\end{align}
%%%%
A partir de las expresiones \eqref{ec_apD:32} y \eqref{ec_apD:34} se determinan las componentes del vector $\mathbf{u}$, que resultan:
%%%%
\begin{subequations}
\label{grup_ec_apD:4}
\begin{align}
{\text{u}}_x &= -\dfrac{\sen\theta '\sen\phi '\cos\phi '\left[\sen\theta '\cos\left(\dfrac{\theta '}{2}\right) - \cos\theta '\sen\left(\dfrac{\theta '}{2}\right)\right]}{\sqrt{1 - \sen^2\theta '\sen^2\phi '}}
\label{ec_apD:35}\\
{\text{u}}_y &= \dfrac{\sen\theta '\cos\theta '\sen\left(\dfrac{\theta '}{2}\right)\sen^2\phi ' - \cos\left(\dfrac{\theta '}{2}\right)\!\left[\sen^2\theta '\cos^2\phi ' + \cos^2\theta '\right]}{\sqrt{1 - \sen^2\theta '\sen^2\phi '}}
\label{ec_apD:36}\\
{\text{u}}_z &= -\dfrac{\cos\theta '\sen\phi '\left[\sen\theta '\cos\left(\dfrac{\theta '}{2}\right) - \cos\theta '\sen\left(\dfrac{\theta '}{2}\right)\right]}{\sqrt{1 - \sen^2\theta '\sen^2\phi '}}
\label{ec_apD:37}
\end{align}
\end{subequations}
%%%%
Trabajando las expresiones \eqref{grup_ec_apD:4} se llega a la expresión de $\mathbf{u}$, que es:
%%%%
\begin{align}
\begin{split}
\mathbf{u} &= -\dfrac{\sen\theta '\sen\left(\dfrac{\theta '}{2}\right)\sen\phi '\cos\phi '}{\sqrt{1 - \sen^2\theta '\sen^2\phi '}}\versor{x} + \dfrac{\cos\left(\dfrac{\theta '}{2}\right)\!\left(\cos\theta '\sen^2\phi ' + \cos^2\phi '\right)}{\sqrt{1 - \sen^2\theta '\sen^2\phi '}}\versor{y}\\
&-\dfrac{\cos\theta '\sen\left(\dfrac{\theta '}{2}\right)\sen\phi '}{\sqrt{1 - \sen^2\theta '\sen^2\phi '}}\versor{z}
\label{ec_apD:38}
\end{split}
\end{align}
%%%%